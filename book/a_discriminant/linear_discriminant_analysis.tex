% The Clever Algorithms Project: http://www.CleverAlgorithms.com
% (c) Copyright 2013 Jason Brownlee. Some Rights Reserved. 
% This work is licensed under a Creative Commons Attribution-Noncommercial-Share Alike 2.5 Australia License.


% Name
% The algorithm name defines the canonical name used to refer to the technique, in addition to common aliases, abbreviations, and acronyms. The name is used in terms of the heading and sub-headings of an algorithm description.
\section{Linear Discriminant Analysis} 
\label{sec:lda}
\index{Linear Discriminant Analysis}
\index{Linear Discriminant Function}
\index{Discriminant Analysis}
\index{LDA}

% other names
% What is the canonical name and common aliases for a technique?
% What are the common abbreviations and acronyms for a technique?
\emph{Linear Discriminant Analysis, Discriminant Analysis, Discriminant Function Analysis, Linear Discriminant Function, LDA.}

% Taxonomy: Lineage and locality
% The algorithm taxonomy defines where a techniques fits into the field, both the specific subfields of Computational Intelligence and Biologically Inspired Computation as well as the broader field of Artificial Intelligence. The taxonomy also provides a context for determining the relation- ships between algorithms. The taxonomy may be described in terms of a series of relationship statements or pictorially as a venn diagram or a graph with hierarchical structure.
\subsection{Taxonomy}
% To what fields of study does a technique belong?
% What are the closely related approaches to a technique?
Linear Discriminant Analysis is a classification method.

Used for classification and for dimensionality reduction.

What is Fisher Linear Discriminant Analysis?

How is it related to Quadratic Discriminant Analysis?
Factor Analysis?
PCA
Regularized Linear Discriminant Analysis?
Kernel Discriminant Analysis?

Related to linear regression which provides a linear model for a numeric rather than categorical.


% Strategy: Problem solving plan
% The strategy is an abstract description of the computational model. The strategy describes the information processing actions a technique shall take in order to achieve an objective. The strategy provides a logical separation between a computational realization (procedure) and a analogous system (metaphor). A given problem solving strategy may be realized as one of a number specific algorithms or problem solving systems. The strategy description is textual using information processing and algorithmic terminology.
\subsection{Strategy}
% What is the information processing objective of a technique?
% What is a techniques plan of action?

used for classification and dimensionality reduction
kind of like regression
linear model for a nominal (categorical) dependent variable.


% help me use this technique
\subsection{Overview}

% what it is good at
\subsubsection{Features}

\begin{itemize}
	\item Used for prediction of a nominal (categorical) dependent variable, cannot be used for regression.
\end{itemize}

% what it is not good at
\subsubsection{Limitations}

\begin{itemize}
	\item 
\end{itemize}


% sample script in R
\subsection{Code Listing}
% listing
Listing~\ref{mass_linear_discriminant_analysis} provides a code listing Linear Discriminant Analysis method in R.
% algorithm and package
The example uses the {lda()} function in the \texttt{MASS} package.
% problem
The test problem is a two-dimensional dataset of 50 samples.

\lstinputlisting[firstline=7,language=r,caption={Example of Linear Discriminant Analysis in R using the \texttt{lda()} function in the \texttt{MASS} package.}, label=mass_linear_discriminant_analysis]{../src/algorithms/discriminant/mass_linear_discriminant_analysis.R}

% other packages
Other packages that provide an implementation of Linear Discriminant Analysis include the \texttt{sparseLDA} which provide sparse LDA, and \texttt{nncRda} which provides regularized LDA.



%see \url{http://www.statmethods.net/advstats/discriminant.html}
%see \url{http://www.r-bloggers.com/an-example-of-linear-discriminant-analysis/}
%see \url{http://little-book-of-r-for-multivariate-analysis.readthedocs.org/en/latest/src/multivariateanalysis.html#linear-discriminant-analysis}

% References: Deeper understanding
% The references element description includes a listing of both primary sources of information about the technique as well as useful introductory sources for novices to gain a deeper understanding of the theory and application of the technique. The description consists of hand-selected reference material including books, peer reviewed conference papers, journal articles, and potentially websites. A bullet-pointed structure is suggested.
\subsection{References}
% What are the primary sources for a technique?
% What are the suggested reference sources for learning more about a technique?

% primary sources
\subsubsection{Primary Sources}



% more info
\subsubsection{More Information}





% END
