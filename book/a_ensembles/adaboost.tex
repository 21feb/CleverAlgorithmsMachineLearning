% The Clever Algorithms Project: http://www.CleverAlgorithms.com
% (c) Copyright 2011 Jason Brownlee. Some Rights Reserved. 
% This work is licensed under a Creative Commons Attribution-Noncommercial-Share Alike 2.5 Australia License.

% Name
% The algorithm name defines the canonical name used to refer to the technique, in addition to common aliases, abbreviations, and acronyms. The name is used in terms of the heading and sub-headings of an algorithm description.
\section{AdaBoost} 
\label{sec:adaboost}
\index{AdaBoost}
\index{AdaBoost.M1}

% other names
% What is the canonical name and common aliases for a technique?
% What are the common abbreviations and acronyms for a technique?
\emph{AdaBoost, AdaBoost.M1}

% Taxonomy: Lineage and locality
% The algorithm taxonomy defines where a techniques fits into the field, both the specific subfields of Computational Intelligence and Biologically Inspired Computation as well as the broader field of Artificial Intelligence. The taxonomy also provides a context for determining the relation- ships between algorithms. The taxonomy may be described in terms of a series of relationship statements or pictorially as a venn diagram or a graph with hierarchical structure.
\subsection{Taxonomy}
% To what fields of study does a technique belong?
% What are the closely related approaches to a technique?
AdaBoost is an Ensemble algorithm.

boosting is sometimes called leverage - leveraging algorithms

AdaBoost.M1 is the classical algorithm
AdaBoost.M2
AdaBoost.R
AdaBoost.R2
Fast AdaBoost


Friends: LPBoost, TotalBoost, BrownBoost, MadaBoost, LogitBoost
AnyBoost framework

% Strategy: Problem solving plan
% The strategy is an abstract description of the computational model. The strategy describes the information processing actions a technique shall take in order to achieve an objective. The strategy provides a logical separation between a computational realization (procedure) and a analogous system (metaphor). A given problem solving strategy may be realized as one of a number specific algorithms or problem solving systems. The strategy description is textual using information processing and algorithmic terminology.
\subsection{Strategy}
% What is the information processing objective of a technique?
% What is a techniques plan of action?

all about one learner learning the weaknesses of another learner - continue until a desired level of performance is achieved

statistical view: stage-wise gradient descent that minimizing an exponetial loss function

adaboost was one of the first boosting algorithms that could adapt its weak learners


% help me use this technique
\subsection{Overview}

% what it is good at
\subsubsection{Features}

\begin{itemize}
	\item It is very difficult for AdaBoost to overfit a training dataset. This feature is highlighted as a limitation of the statistical view of the approach given that it cannot explain how this is the case \cite{Mease2008}.
\end{itemize}

% what it is not good at
\subsubsection{Limitations}

\begin{itemize}
	\item Cannot handle noisy data? \cite{Long2010}
\end{itemize}


% sample script in R
\subsection{Code Listing}
% listing
Listing~\ref{ada_adaboost} provides a code listing of the AdaBoost algorithm in R.
% algorithm and package
The example uses the \texttt{ada()} function in the \texttt{ada} package \cite{Culp2007}. In addition to AdaBoost, the \texttt{ada} package also provides Gentle Adaboost and Real Adaboost.
% TODO more about the package and the functions capabilities
Culp, et al. also provide a vignette of the \texttt{ada} package with more examples and references \cite{Culp2006}.

% problem
The test problem is a three-dimensional dataset, where the $x$, $y$ and $z$ attributes are numerical and drawn from normal distributions with different means. The objective is to predict the categorical dependent $z$ given the independents $x$ and $y$.

% code listing
\lstinputlisting[firstline=7,language=r,caption={Example of the Adaboost algorithm in R using the \texttt{ada()} function of the \texttt{ada} package.}, label=ada_adaboost]{../src/algorithms/ensemble/ada_adaboost.R}

% other packages
Other packages that provide implementations of the AdaBoost algorithm include the \texttt{adabag} package that provides the Adaboost.M1 algorithm \cite{Cortes2011}, and the texttt{gbm} package implements the Adaboost algorithm with an exponential loss function and uses gradient descent during training \cite{Ridgeway2007a}.

% References: Deeper understanding
% The references element description includes a listing of both primary sources of information about the technique as well as useful introductory sources for novices to gain a deeper understanding of the theory and application of the technique. The description consists of hand-selected reference material including books, peer reviewed conference papers, journal articles, and potentially websites. A bullet-pointed structure is suggested.
\subsection{References}
% What are the primary sources for a technique?
% What are the suggested reference sources for learning more about a technique?

% primary sources
\subsubsection{Primary Sources}

Proposed by Freund and Schapire \cite{Freund1997}.

% more info
\subsubsection{More Information}


Introduction to boosting \cite{Freund1999} (maybe in the chapter overview...)

very detailed introduction to boosting \cite{Meir2003}

% END
