% The Clever Algorithms Project: http://www.CleverAlgorithms.com
% (c) Copyright 2013 Jason Brownlee. Some Rights Reserved. 
% This work is licensed under a Creative Commons Attribution-Noncommercial-Share Alike 2.5 Australia License.


% Name
% The algorithm name defines the canonical name used to refer to the technique, in addition to common aliases, abbreviations, and acronyms. The name is used in terms of the heading and sub-headings of an algorithm description.
\section{Gradient Boosting} 
\label{sec:gradientboosting}
\index{Gradient Boosting}

% other names
% What is the canonical name and common aliases for a technique?
% What are the common abbreviations and acronyms for a technique?
\emph{Gradient Boosting, Gradient Boosting Machine, GBM}

% Taxonomy: Lineage and locality
% The algorithm taxonomy defines where a techniques fits into the field, both the specific subfields of Computational Intelligence and Biologically Inspired Computation as well as the broader field of Artificial Intelligence. The taxonomy also provides a context for determining the relation- ships between algorithms. The taxonomy may be described in terms of a series of relationship statements or pictorially as a venn diagram or a graph with hierarchical structure.
\subsection{Taxonomy}
% To what fields of study does a technique belong?
% What are the closely related approaches to a technique?
Gradient Boosting is an Ensemble algorithm.

Gradient Boosted Decision Trees (GBDT)
Gradient Boosted Regression Trees (GBRT) - application of the method to regression trees
Stochastic Gradient Boosting
Gradient Boost
Gradient Tree Boosting
TreeBoost
Generalized Boosting Model
Functional Gradient Boosting
Gradient Boosted Models
Multiple Additive Regression Trees (MART) - commercial
TreeNet - commercial


% Strategy: Problem solving plan
% The strategy is an abstract description of the computational model. The strategy describes the information processing actions a technique shall take in order to achieve an objective. The strategy provides a logical separation between a computational realization (procedure) and a analogous system (metaphor). A given problem solving strategy may be realized as one of a number specific algorithms or problem solving systems. The strategy description is textual using information processing and algorithmic terminology.
\subsection{Strategy}
% What is the information processing objective of a technique?
% What is a techniques plan of action?


for regression only?



% help me use this technique
\subsection{Overview}

% what it is good at
\subsubsection{Features}

\begin{itemize}
	\item 
\end{itemize}

% what it is not good at
\subsubsection{Limitations}

\begin{itemize}
	\item 
\end{itemize}


% sample script in R
\subsection{Code Listing}
% listing
Listing~\ref{gbm_gradient_boosting} provides a code listing of the Gradient Boosting algorithm in R.
% algorithm and package
The example uses the \texttt{gbm()} function in the \texttt{gbm} package \cite{Ridgeway2007a}. The \texttt{gbm()} function makes use of the \texttt{rpart()} function from the \texttt{rpart} package to create an ensemble of trees as the weak learners.
% TODO more about the package and the functions capabilities
Ridgeway also provides a vignette of the \texttt{gbm} package with more examples and references \cite{Ridgeway2007}.

% problem
The test problem is a three-dimensional dataset, where the $x$, $y$ and $z$ attributes are numerical and drawn from normal distributions with different means. The objective is to predict the categorical dependent $z$ given the independents $x$ and $y$.

% code listing
\lstinputlisting[firstline=7,language=r,caption={Example of the Gradient Boosting algorithm in R using the \texttt{gbm()} function of the \texttt{gbm} package.}, label=gbm_gradient_boosting]{../src/algorithms/ensemble/gbm_gradient_boosting.R}

% other packages


% References: Deeper understanding
% The references element description includes a listing of both primary sources of information about the technique as well as useful introductory sources for novices to gain a deeper understanding of the theory and application of the technique. The description consists of hand-selected reference material including books, peer reviewed conference papers, journal articles, and potentially websites. A bullet-pointed structure is suggested.
\subsection{References}
% What are the primary sources for a technique?
% What are the suggested reference sources for learning more about a technique?

% primary sources
\subsubsection{Primary Sources}

Gradient Boosting was proposed by Friedman \cite{Friedman2001}.

% more info
\subsubsection{More Information}



% END
