% The Clever Algorithms Project: http://www.CleverAlgorithms.com
% (c) Copyright 2011 Jason Brownlee. Some Rights Reserved. 
% This work is licensed under a Creative Commons Attribution-Noncommercial-Share Alike 2.5 Australia License.

% Name
% The algorithm name defines the canonical name used to refer to the technique, in addition to common aliases, abbreviations, and acronyms. The name is used in terms of the heading and sub-headings of an algorithm description.
\section{Gradient Descent} 
\label{sec:gradient_descent}
\index{Gradient Descent}
\index{Steepest Descent Method}

% other names
% What is the canonical name and common aliases for a technique?
% What are the common abbreviations and acronyms for a technique?
\emph{Gradient Descent, Steepest Descent Method.}

% Taxonomy: Lineage and locality
\subsection{Taxonomy}
% To what fields of study does a technique belong?
Gradient Descent is an optimization method for unconstrained nonlinear optimization.
It is a direct search method that does require a differentiable function for a gradient, instead it infers the gradient from sampling.
% What are the closely related approaches to a technique? 
Stochastic Gradient Descent (Online Gradient Descent) is an extension commonly used to train a classification or regression model where the gradient is a cost or loss function and the next step is recomputed with each new observation or defined set of observations.


% Strategy: Problem solving plan
% The strategy is an abstract description of the computational model. The strategy describes the information processing actions a technique shall take in order to achieve an objective. The strategy provides a logical separation between a computational realization (procedure) and a analogous system (metaphor). A given problem solving strategy may be realized as one of a number specific algorithms or problem solving systems. The strategy description is textual using information processing and algorithmic terminology.
\subsection{Strategy}
% What is the information processing objective of a technique?
% What is a techniques plan of action?


% help me use this technique
\subsection{Overview}

% what it is good at
\subsubsection{Features}

\begin{itemize}
	\item It is relatively simple to understand and implement.
	\item Works well non-differentiable functions, discontinuous functions, non-smooth functions, and noisy functions.
\end{itemize}

% what it is not good at
\subsubsection{Limitations}

\begin{itemize}
	\item It is consider slow relative to modern methods.
\end{itemize}

% sample script in R
\subsection{Code Listing}

% other packages
Some other packages that provide an implementation of Gradient Descent include \texttt{animation}.


% TODO look at the gsl package that makes use of the Gnu Scientific Library multimin


% References: Deeper understanding
% The references element description includes a listing of both primary sources of information about the technique as well as useful introductory sources for novices to gain a deeper understanding of the theory and application of the technique. The description consists of hand-selected reference material including books, peer reviewed conference papers, journal articles, and potentially websites. A bullet-pointed structure is suggested.
\subsection{References}
% What are the primary sources for a technique?
% What are the suggested reference sources for learning more about a technique?

% primary sources
\subsubsection{Primary Sources}


% more info
\subsubsection{More Information}




% END
