% The Clever Algorithms Project: http://www.CleverAlgorithms.com
% (c) Copyright 2011 Jason Brownlee. Some Rights Reserved. 
% This work is licensed under a Creative Commons Attribution-Noncommercial-Share Alike 2.5 Australia License.

% Name
% The algorithm name defines the canonical name used to refer to the technique, in addition to common aliases, abbreviations, and acronyms. The name is used in terms of the heading and sub-headings of an algorithm description.
\section{Nelder-Mead Method} 
\label{sec:neldermead}
\index{Nelder-Mead Method}
\index{Simplex Method}

% other names
% What is the canonical name and common aliases for a technique?
% What are the common abbreviations and acronyms for a technique?
\emph{Nelder-Mead Method, Downhill Simplex Method, Simplex Method.}

% Taxonomy: Lineage and locality
\subsection{Taxonomy}
% To what fields of study does a technique belong?
% What are the closely related approaches to a technique?
Nelder-Mead Method is an optimization method.

Direct Search method.

% Strategy: Problem solving plan
% The strategy is an abstract description of the computational model. The strategy describes the information processing actions a technique shall take in order to achieve an objective. The strategy provides a logical separation between a computational realization (procedure) and a analogous system (metaphor). A given problem solving strategy may be realized as one of a number specific algorithms or problem solving systems. The strategy description is textual using information processing and algorithmic terminology.
\subsection{Strategy}
% What is the information processing objective of a technique?
% What is a techniques plan of action?


% help me use this technique
\subsection{Overview}

% what it is good at
\subsubsection{Features}

\begin{itemize}
	\item Suitable for convex response surfaces.
\end{itemize}

% what it is not good at
\subsubsection{Limitations}

\begin{itemize}
	\item An optimal value for the $\alpha$ parameter can be difficult to find.
\end{itemize}


% sample script in R
\subsection{Code Listing}
Listing~\ref{stat_nelder_mead} provides a code listing of the Nelder-Mead method in R using the \texttt{constrOptim()} function in the \texttt{stats} core package. 

% problem
The test problem is the Rosenbrock function in two-dimensions where $x_i\in[-2.048,2.048]$ and the optimum is at $x_i=1$.

\lstinputlisting[firstline=7,language=r,caption={Example of Nelder-Mead in R using the \texttt{constrOptim()} function in the \texttt{stats} core packag.}, label=stat_nelder_mead]{../src/algorithms/optimization/stat_nelder_mead.R}

% References: Deeper understanding
% The references element description includes a listing of both primary sources of information about the technique as well as useful introductory sources for novices to gain a deeper understanding of the theory and application of the technique. The description consists of hand-selected reference material including books, peer reviewed conference papers, journal articles, and potentially websites. A bullet-pointed structure is suggested.
\subsection{References}
% What are the primary sources for a technique?
% What are the suggested reference sources for learning more about a technique?

% primary sources
\subsubsection{Primary Sources}

seminal \cite{Nelder1965}

% more info
\subsubsection{More Information}




% END
