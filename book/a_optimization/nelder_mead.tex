% The Clever Algorithms Project: http://www.CleverAlgorithms.com
% (c) Copyright 2011 Jason Brownlee. Some Rights Reserved. 
% This work is licensed under a Creative Commons Attribution-Noncommercial-Share Alike 2.5 Australia License.

% Name
% The algorithm name defines the canonical name used to refer to the technique, in addition to common aliases, abbreviations, and acronyms. The name is used in terms of the heading and sub-headings of an algorithm description.
\section{Nelder-Mead Method} 
\label{sec:neldermead}
\index{Nelder-Mead Method}
\index{Simplex Algorithm}
\index{Amoeba Algorithm}

% other names
% What is the canonical name and common aliases for a technique?
% What are the common abbreviations and acronyms for a technique?
\emph{Nelder-Mead Method, Downhill Simplex Method, Simplex Method.}

% Taxonomy: Lineage and locality
\subsection{Taxonomy}
% To what fields of study does a technique belong?
% What are the closely related approaches to a technique?
Nelder-Mead Method is an optimization method for multidimensional nonlinear unconstrained functions.
It is a direct search method that does require a differentiable function for a gradient, instead it infers the gradient from sampling.

% Strategy: Problem solving plan
% The strategy is an abstract description of the computational model. The strategy describes the information processing actions a technique shall take in order to achieve an objective. The strategy provides a logical separation between a computational realization (procedure) and a analogous system (metaphor). A given problem solving strategy may be realized as one of a number specific algorithms or problem solving systems. The strategy description is textual using information processing and algorithmic terminology.
\subsection{Strategy}
% What is the information processing objective of a technique?
Nelder-Mead optimizes a function by overlaying a simplex (geometrical pattern) in the domain and iteratively increasing and/or reducing its size until an optimal value is found.
% What is a techniques plan of action?

% help me use this technique
\subsection{Overview}

% what it is good at
\subsubsection{Features}

\begin{itemize}
	\item It is relatively simple to implement and fast to compute.
	\item It can handle a discontinuous, noisy and/or non-smooth response surface given that it does not rely on an explicit derivative.
\end{itemize}

% what it is not good at
\subsubsection{Limitations}

\begin{itemize}
	\item It is dependent on the starting position and call be caught by local optima in multimodal functions.
\end{itemize}

% sample script in R
\subsection{Code Listing}
% listing
Listing~\ref{stats_nelder_mead} provides a code listing Nelder-Mead method in R solving a two-dimensional nonlinear unconstrained optimization function.
% algorithm and package
The example uses the {optim()} function in the \texttt{stats} core package configured to use the ``Nelder-Mead'' method. 
% problem
The test problem is the Rosenbrock function in two-dimensions where the optimum is at $x=1, y=1$. The starting position for the algorithm is taken as a random point $x,y \in [-3,3]$.

\lstinputlisting[firstline=7,language=r,caption={Example of Nelder-Mead in R using the \texttt{optim()} function in the \texttt{stats} core package.}, label=stats_nelder_mead]{../src/algorithms/optimization/stats_nelder_mead.R}

% other packages
Some other packages that provide an implementation of the Nelder-Mead algorithm include \texttt{gsl} and \texttt{neldermead}.

% References: Deeper understanding
% The references element description includes a listing of both primary sources of information about the technique as well as useful introductory sources for novices to gain a deeper understanding of the theory and application of the technique. The description consists of hand-selected reference material including books, peer reviewed conference papers, journal articles, and potentially websites. A bullet-pointed structure is suggested.
\subsection{References}
% What are the primary sources for a technique?
% What are the suggested reference sources for learning more about a technique?

% primary sources
\subsubsection{Primary Sources}
% seminal
The Nelder-Mead method was propose by Nelder and Mead in 1965 \cite{Nelder1965} as an extension of Spendley, et al. direct search method \cite{Spendley1962}.

% more info
\subsubsection{More Information}
% study
Lagarias, et al provide a investigation of the convergence properties of the Nelder-Mead method on convex one- and two-dimensional problems \cite{Lagarias1998}.
% book
Walters provides a book that focuses on the method which he refers to as Sequential Simplex Optimization \cite{Walters1991}.

% END
