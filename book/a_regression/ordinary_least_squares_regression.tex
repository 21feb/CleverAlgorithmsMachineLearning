% The Clever Algorithms Project: http://www.CleverAlgorithms.com
% (c) Copyright 2011 Jason Brownlee. Some Rights Reserved. 
% This work is licensed under a Creative Commons Attribution-Noncommercial-Share Alike 2.5 Australia License.

% Name
% The algorithm name defines the canonical name used to refer to the technique, in addition to common aliases, abbreviations, and acronyms. The name is used in terms of the heading and sub-headings of an algorithm description.
\section{Ordinary Least Squares Regression} 
\label{sec:ordinary}
\index{Ordinary Least Squares Regression}
\index{Ordinary Linear Regression}
\index{Simple Linear Regression}
\index{Linear Least Squares}

% other names
% What is the canonical name and common aliases for a technique?
% What are the common abbreviations and acronyms for a technique?
\emph{Ordinary Least Squares Regression, Ordinary Linear Regression, Simple Linear Regression, Linear Least Squares}

% Taxonomy: Lineage and locality
% The algorithm taxonomy defines where a techniques fits into the field, both the specific subfields of Computational Intelligence and Biologically Inspired Computation as well as the broader field of Artificial Intelligence. The taxonomy also provides a context for determining the relation- ships between algorithms. The taxonomy may be described in terms of a series of relationship statements or pictorially as a venn diagram or a graph with hierarchical structure.
\subsection{Taxonomy}
% To what fields of study does a technique belong?
% What are the closely related approaches to a technique?
Ordinary Least Squares is a regression method.


% Strategy: Problem solving plan
% The strategy is an abstract description of the computational model. The strategy describes the information processing actions a technique shall take in order to achieve an objective. The strategy provides a logical separation between a computational realization (procedure) and a analogous system (metaphor). A given problem solving strategy may be realized as one of a number specific algorithms or problem solving systems. The strategy description is textual using information processing and algorithmic terminology.
\subsection{Strategy}
% What is the information processing objective of a technique?
% What is a techniques plan of action?

Fit a straight line to a set of data points
Assumes attributes are drawn from a normal distribution and a straight line can be drawn through the mean.

minimize sum of squared residuals


% help me use this technique
\subsection{Overview}

% what it is good at
\subsubsection{Features}

\begin{itemize}
	\item Simple method for feature selection.
	\item The dependent variable must be continuous, but the independent variables may be continuous or categorical.
\end{itemize}

% what it is not good at
\subsubsection{Limitations}

\begin{itemize}
	\item 
\end{itemize}

% sample script in R
\subsection{Code Listing}
% listing
Listing~\ref{stats_ordinary_least_squares} provides a code listing Ordinary Least Squares method in R to find a line of best fit for a two-dimensional data set.
% algorithm and package
The example uses the {lm()} function in the \texttt{stats} core package which is responsible for fitting linear models.
% problem
The test problem is a two-dimensional dataset of 50 samples, where the x-values are drawn from a uniformly random distribution $x \in [0,10]$ and y values are the x value plus a value drawn from a normally random 
distribution with a mean of 0 and a standard deviation of 1.

% what solving method does lm use?
% classification example?

\lstinputlisting[firstline=7,language=r,caption={Example of Ordinary Least Squares in R using the \texttt{lm()} function in the \texttt{stats} core package.}, label=stats_ordinary_least_squares]{../src/algorithms/regression/stats_ordinary_least_squares.R}

% other packages ?


% References: Deeper understanding
% The references element description includes a listing of both primary sources of information about the technique as well as useful introductory sources for novices to gain a deeper understanding of the theory and application of the technique. The description consists of hand-selected reference material including books, peer reviewed conference papers, journal articles, and potentially websites. A bullet-pointed structure is suggested.
\subsection{References}
% What are the primary sources for a technique?
% What are the suggested reference sources for learning more about a technique?

% primary sources
\subsubsection{Primary Sources}



% more info
\subsubsection{More Information}

see \url{http://en.wikipedia.org/wiki/Ordinary_least_squares}
see \url{http://en.wikibooks.org/wiki/R_Programming/Linear_Models}
see \url{http://www.mayin.org/ajayshah/KB/R/html/o1.html}
see \url{http://cran.r-project.org/doc/contrib/Faraway-PRA.pdf}

% END
