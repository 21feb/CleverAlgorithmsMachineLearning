% The Clever Algorithms Project: http://www.CleverAlgorithms.com
% (c) Copyright 2011 Jason Brownlee. Some Rights Reserved. 
% This work is licensed under a Creative Commons Attribution-Noncommercial-Share Alike 2.5 Australia License.

% Name
% The algorithm name defines the canonical name used to refer to the technique, in addition to common aliases, abbreviations, and acronyms. The name is used in terms of the heading and sub-headings of an algorithm description.
\section{LASSO} 
\label{sec:lasso}
\index{LASSO}
\index{Least Absolute Shrinkage and Selection Operator}

% other names
% What is the canonical name and common aliases for a technique?
% What are the common abbreviations and acronyms for a technique?
\emph{LASSO, Least Absolute Shrinkage and Selection Operator}

% Taxonomy: Lineage and locality
% The algorithm taxonomy defines where a techniques fits into the field, both the specific subfields of Computational Intelligence and Biologically Inspired Computation as well as the broader field of Artificial Intelligence. The taxonomy also provides a context for determining the relation- ships between algorithms. The taxonomy may be described in terms of a series of relationship statements or pictorially as a venn diagram or a graph with hierarchical structure.
\subsection{Taxonomy}
% To what fields of study does a technique belong?
% What are the closely related approaches to a technique?
LASSO is a Regularization algorithm.
model selection technique

Least Angle Regression (LAR) algorithm for solving the Lasso

grouped lasso is an extension?
Dantzig selector is an extension?
elastic net is an extension?
generalized elastic net is an extension?
graphical lasso is an extension?




% Strategy: Problem solving plan
% The strategy is an abstract description of the computational model. The strategy describes the information processing actions a technique shall take in order to achieve an objective. The strategy provides a logical separation between a computational realization (procedure) and a analogous system (metaphor). A given problem solving strategy may be realized as one of a number specific algorithms or problem solving systems. The strategy description is textual using information processing and algorithmic terminology.
\subsection{Strategy}
% What is the information processing objective of a technique?
% What is a techniques plan of action?

penalized least squares

lasso is for linear regression models?

LARS algorithm

uses the L1-penalty, the lasso does both continuous shrinkage and automatic variable selection simultaneously

imposes a bound on the absolute sum of the coefficients

\subsection{Overview}

% what it is good at
\subsubsection{Features}

\begin{itemize}
	\item results in a sparse representation (fewer variables)
\end{itemize}

% what it is not good at
\subsubsection{Limitations}

\cite{Zou2005} mentions 3 limitations of LASSO

\begin{itemize}
	\item not good at p>>n (more attributes than instances)
\end{itemize}


% sample script in R
\subsection{Code Listing}
% listing
Listing~\ref{lars_lasso_regression} provides a code listing LASSO method in R.
% algorithm and package
The example uses the \texttt{lars()} function in the \texttt{lars} core package. The \texttt{lars} package provides the Least Angle Regression, Lasso, and the Forward Stagewise algorithms for regularization \cite{Hastie2011}.
% problem

http://www.utstat.utoronto.ca/reid/sta414/Table33R.txt

% TODO provide a better problem with real variable selection

\lstinputlisting[firstline=7,language=r,caption={Example of LASSO in R using the \texttt{lars()} function in the \texttt{lars} package.}, label=lars_lasso_regression]{../src/algorithms/regularization/lars_lasso_regression.R}

% other packages
Other packages provide implementations of the Lasso penalty method.
The \texttt{glmnet} package provides the Lasso and elastic-net regularization for generalized linear models \cite{Friedman2011}.
The \texttt{grplasso} package provides an the Group Lasso penalty method for fitting models \cite{Meier2009}.
The \texttt{grpreg} package provides lasso regularization with grouped covariates \cite{Brehen2011}.

% lmmlasso package?!?
% penalized package

% References: Deeper understanding
% The references element description includes a listing of both primary sources of information about the technique as well as useful introductory sources for novices to gain a deeper understanding of the theory and application of the technique. The description consists of hand-selected reference material including books, peer reviewed conference papers, journal articles, and potentially websites. A bullet-pointed structure is suggested.
\subsection{References}
% What are the primary sources for a technique?
% What are the suggested reference sources for learning more about a technique?

% primary sources
\subsubsection{Primary Sources}

Proposed by Tibshirani \cite{Tibshirani1996}.

% more info
\subsubsection{More Information}



% END
