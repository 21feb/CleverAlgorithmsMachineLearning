% The Clever Algorithms Project: http://www.CleverAlgorithms.com
% (c) Copyright 2011 Jason Brownlee. Some Rights Reserved. 
% This work is licensed under a Creative Commons Attribution-Noncommercial-Share Alike 2.5 Australia License.

% Name
% The algorithm name defines the canonical name used to refer to the technique, in addition to common aliases, abbreviations, and acronyms. The name is used in terms of the heading and sub-headings of an algorithm description.
\section{Ridge Regression} 
\label{sec:ridge}
\index{Ridge Regression}
\index{Tikhonov Regularization}
\index{Tikhonov-Miller Method}
\index{Phillips-Twomey Method}
\index{Constrained Linear Inversion}
\index{Linear Regularization}

% other names
% What is the canonical name and common aliases for a technique?
% What are the common abbreviations and acronyms for a technique?
\emph{Ridge Regression, Tikhonov Regularization, Tikhonov-Miller Method, Phillips-Twomey Method, Constrained Linear Inversion, Linear Regularization}

% Taxonomy: Lineage and locality
% The algorithm taxonomy defines where a techniques fits into the field, both the specific subfields of Computational Intelligence and Biologically Inspired Computation as well as the broader field of Artificial Intelligence. The taxonomy also provides a context for determining the relation- ships between algorithms. The taxonomy may be described in terms of a series of relationship statements or pictorially as a venn diagram or a graph with hierarchical structure.
\subsection{Taxonomy}
% To what fields of study does a technique belong?
% What are the closely related approaches to a technique?
Ridge Regression is a Regression algorithm.

It is a linear regularization method.
used with least squares regression and logistic regression !?


Related to Bridge Regression??? 

% Strategy: Problem solving plan
% The strategy is an abstract description of the computational model. The strategy describes the information processing actions a technique shall take in order to achieve an objective. The strategy provides a logical separation between a computational realization (procedure) and a analogous system (metaphor). A given problem solving strategy may be realized as one of a number specific algorithms or problem solving systems. The strategy description is textual using information processing and algorithmic terminology.
\subsection{Strategy}
% What is the information processing objective of a technique?
% What is a techniques plan of action?

used for Regularization
find coefficients via optimization - unconstrained multipler
assumes that coefficients after normalization are not very large

Bayesian understanding of the method
Restricted Least Squares understanding of the method 

see \url{http://cran.r-project.org/doc/contrib/Faraway-PRA.pdf}

minimizes the residual sum of squares subject to a bound on the L2-norm of the coefficients
better performance than OLS via a bias–variance trade-off



% help me use this technique
\subsection{Overview}

% what it is good at
\subsubsection{Features}

\begin{itemize}
	\item Ridge regression result in small coefficients which may be considered a less complex model.
	\item The method is appropriate when other methods (such as least squares) appear to be unstable.
	\item Appropriate when the design matrix is collinear. (?)
	\item The ridge constant $\lambda$ is typically selected in the range $[0,1]$, where $\lambda = 0$ corresponds to a least squares regression.
\end{itemize}

% what it is not good at
\subsubsection{Limitations}

\begin{itemize}
	\item The coefficients prepared by Ridge Regression is biased. The cost function is the square of the bias plus the variance. An improvement can be achieved by a large drop in variance at the expense of bias.
	\item keeps all predictors
\end{itemize}


% sample script in R
\subsection{Code Listing}
% listing
Listing~\ref{mass_ridge_regression} provides a code listing Ridge Regression method in R to find a line of best fit for a two-dimensional data set.
% algorithm and package
The example uses the \texttt{lm.ridge()} function in the \texttt{MASS} core package which is responsible for fitting linear models using Ridge Regression.
% problem
%The test problem is a two-dimensional dataset of 50 samples, where the x-values are drawn from a uniformly random distribution $x \in [0,10]$ and y values are the x value plus a value drawn from a normally random distribution with a mean of 0 and a standard deviation of 1.

% TODO how do make use of a resulting lambda?
% TODO how can you make use of a resulting model to make predictions?


\lstinputlisting[firstline=7,language=r,caption={Example of Ridge Regression in R using the \texttt{lm.ridge()} function in the \texttt{MASS} package.}, label=mass_ridge_regression]{../src/algorithms/regularization/mass_ridge_regression.R}

% other packages ?
Other packages that provide an implementation of Ridge Regression include \texttt{parcor}.


% References: Deeper understanding
% The references element description includes a listing of both primary sources of information about the technique as well as useful introductory sources for novices to gain a deeper understanding of the theory and application of the technique. The description consists of hand-selected reference material including books, peer reviewed conference papers, journal articles, and potentially websites. A bullet-pointed structure is suggested.
\subsection{References}
% What are the primary sources for a technique?
% What are the suggested reference sources for learning more about a technique?

% primary sources
\subsubsection{Primary Sources}


% more info
\subsubsection{More Information}



% END
