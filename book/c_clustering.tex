% The Clever Algorithms Project: http://www.CleverAlgorithms.com
% (c) Copyright 2013 Jason Brownlee. Some Rights Reserved. 
% This work is licensed under a Creative Commons Attribution-Noncommercial-Share Alike 2.5 Australia License.


% This is a chapter

\renewcommand{\bibsection}{\subsection{\bibname}}
\begin{bibunit}

\chapter{Clustering and Mixture Models}
\label{ch:clustering}
\index{Cluster Analysis}
\index{Mixture Models}

\section{Overview}
This chapter describes Clustering and Mixture Models.


% Strategy: Problem solving plan
% The strategy is an abstract description of the computational model. The strategy describes the information processing actions a technique shall take in order to achieve an objective. The strategy provides a logical separation between a computational realization (procedure) and a analogous system (metaphor). A given problem solving strategy may be realized as one of a number specific algorithms or problem solving systems. The strategy description is textual using information processing and algorithmic terminology.
\subsection{Strategy}
% What is the information processing objective of a technique?
% What is a techniques plan of action?

Clustering and Mixture Models


% Heuristics: Usage guidelines
% The heuristics element describe the commonsense, best practice, and demonstrated rules for applying and configuring a parameterized algorithm. The heuristics relate to the technical details of the techniques procedure and data structures for general classes of application (neither specific implementations not specific problem instances). The heuristics are described textually, such as a series of guidelines in a bullet-point structure.
\subsection{Heuristics}
% What are the suggested configurations for a technique?
% What are the guidelines for the application of a technique to a problem instance?

\begin{itemize}
	\item 
\end{itemize}



% References: Deeper understanding
% The references element description includes a listing of both primary sources of information about the technique as well as useful introductory sources for novices to gain a deeper understanding of the theory and application of the technique. The description consists of hand-selected reference material including books, peer reviewed conference papers, journal articles, and potentially websites. A bullet-pointed structure is suggested.
\subsection{References}
% What are the primary sources for a technique?
% What are the suggested reference sources for learning more about a technique?

% primary sources
\subsubsection{Primary Sources}


% more info
\subsubsection{More Information}

see clustering view \url{http://cran.r-project.org/web/views/Cluster.html}

\putbib
\end{bibunit}


\newpage\begin{bibunit}% The Clever Algorithms Project: http://www.CleverAlgorithms.com
% (c) Copyright 2013 Jason Brownlee. Some Rights Reserved. 
% This work is licensed under a Creative Commons Attribution-Noncommercial-Share Alike 2.5 Australia License.


% Name
% The algorithm name defines the canonical name used to refer to the technique, in addition to common aliases, abbreviations, and acronyms. The name is used in terms of the heading and sub-headings of an algorithm description.
\section{K-means} 
\label{sec:kmeans}
\index{K-means}


% other names
% What is the canonical name and common aliases for a technique?
% What are the common abbreviations and acronyms for a technique?
\emph{K-means, Kmeans.}

% Taxonomy: Lineage and locality
% The algorithm taxonomy defines where a techniques fits into the field, both the specific subfields of Computational Intelligence and Biologically Inspired Computation as well as the broader field of Artificial Intelligence. The taxonomy also provides a context for determining the relation- ships between algorithms. The taxonomy may be described in terms of a series of relationship statements or pictorially as a venn diagram or a graph with hierarchical structure.
\subsection{Taxonomy}
% To what fields of study does a technique belong?
% What are the closely related approaches to a technique?
K-means is a clustering algorithm.

% Strategy: Problem solving plan
% The strategy is an abstract description of the computational model. The strategy describes the information processing actions a technique shall take in order to achieve an objective. The strategy provides a logical separation between a computational realization (procedure) and a analogous system (metaphor). A given problem solving strategy may be realized as one of a number specific algorithms or problem solving systems. The strategy description is textual using information processing and algorithmic terminology.
\subsection{Strategy}
% What is the information processing objective of a technique?
% What is a techniques plan of action?



% help me use this technique
\subsection{Overview}

% what it is good at
\subsubsection{Features}

\begin{itemize}
	\item 
\end{itemize}

% what it is not good at
\subsubsection{Limitations}

\begin{itemize}
	\item 
\end{itemize}


% sample script in R
\subsection{Code Listing}
% listing
Listing~\ref{stats_kmeans} provides a code listing of the K-Means clustering method in R to identify cluster centres in a two-dimensional dataset.
% algorithm and package
The example uses the \texttt{kmeans()} function in the \texttt{stats} core package.
% TODO more about the package and the functions capabilities

% problem
The test problem is a two-dimensional dataset, where the $x$ and $y$ attributes are numerical and drawn from normal distributions around 0 and 4. 

% code listing
\lstinputlisting[firstline=7,language=r,caption={Example of K-Means clustering in R using the \texttt{kmeans()} function of the \texttt{stats} core package.}, label=stats_kmeans]{../src/algorithms/clustering/stats_kmeans.R}

% other packages



% References: Deeper understanding
% The references element description includes a listing of both primary sources of information about the technique as well as useful introductory sources for novices to gain a deeper understanding of the theory and application of the technique. The description consists of hand-selected reference material including books, peer reviewed conference papers, journal articles, and potentially websites. A bullet-pointed structure is suggested.
\subsection{References}
% What are the primary sources for a technique?
% What are the suggested reference sources for learning more about a technique?

% primary sources
\subsubsection{Primary Sources}



% more info
\subsubsection{More Information}





% END
\putbib\end{bibunit}
\newpage\begin{bibunit}% The Clever Algorithms Project: http://www.CleverAlgorithms.com
% (c) Copyright 2011 Jason Brownlee. Some Rights Reserved. 
% This work is licensed under a Creative Commons Attribution-Noncommercial-Share Alike 2.5 Australia License.

% Name
% The algorithm name defines the canonical name used to refer to the technique, in addition to common aliases, abbreviations, and acronyms. The name is used in terms of the heading and sub-headings of an algorithm description.
\section{Expectation Maximization} 
\label{sec:em}
\index{Expectation Maximization}
\index{EM}


% other names
% What is the canonical name and common aliases for a technique?
% What are the common abbreviations and acronyms for a technique?
\emph{Expectation Maximization, EM.}

% Taxonomy: Lineage and locality
% The algorithm taxonomy defines where a techniques fits into the field, both the specific subfields of Computational Intelligence and Biologically Inspired Computation as well as the broader field of Artificial Intelligence. The taxonomy also provides a context for determining the relation- ships between algorithms. The taxonomy may be described in terms of a series of relationship statements or pictorially as a venn diagram or a graph with hierarchical structure.
\subsection{Taxonomy}
% To what fields of study does a technique belong?
% What are the closely related approaches to a technique?
Expectation Maximization is a Mixture Model.

% Strategy: Problem solving plan
% The strategy is an abstract description of the computational model. The strategy describes the information processing actions a technique shall take in order to achieve an objective. The strategy provides a logical separation between a computational realization (procedure) and a analogous system (metaphor). A given problem solving strategy may be realized as one of a number specific algorithms or problem solving systems. The strategy description is textual using information processing and algorithmic terminology.
\subsection{Strategy}
% What is the information processing objective of a technique?
% What is a techniques plan of action?

EM is an algorithm, but it is almost always used to solve specific problems...

% help me use this technique
\subsection{Overview}

% what it is good at
\subsubsection{Features}

\begin{itemize}
	\item 
\end{itemize}

% what it is not good at
\subsubsection{Limitations}

\begin{itemize}
	\item 
\end{itemize}


% sample script in R
\subsection{Code Listing}
% listing
Listing~\ref{mclust_expectation_maximization} provides a code listing of the Expectation Maximization clustering method in R to identify cluster centres in a two-dimensional dataset.
% algorithm and package
The example uses the \texttt{Mclust()} function in the \texttt{mclust}  package.
% TODO more about the package and the functions capabilities

% problem
The test problem is a two-dimensional dataset, where the $x$ and $y$ attributes are numerical and drawn from normal distributions around 0 and 4. 

% code listing
\lstinputlisting[firstline=7,language=r,caption={Example of Expectation Maximization clustering in R using the \texttt{Mclust()} function of the \texttt{mclust} package.}, label=mclust_expectation_maximization]{../src/algorithms/clustering/mclust_expectation_maximization.R}

% other packages


% References: Deeper understanding
% The references element description includes a listing of both primary sources of information about the technique as well as useful introductory sources for novices to gain a deeper understanding of the theory and application of the technique. The description consists of hand-selected reference material including books, peer reviewed conference papers, journal articles, and potentially websites. A bullet-pointed structure is suggested.
\subsection{References}
% What are the primary sources for a technique?
% What are the suggested reference sources for learning more about a technique?

% primary sources
\subsubsection{Primary Sources}



% more info
\subsubsection{More Information}





% END
\putbib\end{bibunit}


