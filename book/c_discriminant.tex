% The Clever Algorithms Project: http://www.CleverAlgorithms.com
% (c) Copyright 2010 Jason Brownlee. Some Rights Reserved. 
% This work is licensed under a Creative Commons Attribution-Noncommercial-Share Alike 2.5 Australia License.

% This is a chapter

\renewcommand{\bibsection}{\subsection{\bibname}}
\begin{bibunit}

\chapter{Discriminant Function Analysis}
\label{ch:discriminant}
\index{Discriminant Function Analysis}

\section{Overview}
This chapter describes Discriminant Function Analysis methods.


% Strategy: Problem solving plan
% The strategy is an abstract description of the computational model. The strategy describes the information processing actions a technique shall take in order to achieve an objective. The strategy provides a logical separation between a computational realization (procedure) and a analogous system (metaphor). A given problem solving strategy may be realized as one of a number specific algorithms or problem solving systems. The strategy description is textual using information processing and algorithmic terminology.
\subsection{Strategy}
% What is the information processing objective of a technique?
% What is a techniques plan of action?

Discriminant Function Analysis


Linear Discriminant Analysis
Flexible Discriminant Analysis
Fisher Linear Discriminant Analysis
Factorial Discriminant Analysis
Mixture Discriminant Analysis
Quadratic Discriminant Analysis
Gaussian Discriminant Analysis
Shrunken Centroids Regularized Discriminant Analysis (rda, sda)
Heteroscedastic Discriminant Analysis (hda)
Stepwise Diagonal Discriminant Analysis (SDDA)

% Heuristics: Usage guidelines
% The heuristics element describe the commonsense, best practice, and demonstrated rules for applying and configuring a parameterized algorithm. The heuristics relate to the technical details of the techniques procedure and data structures for general classes of application (neither specific implementations not specific problem instances). The heuristics are described textually, such as a series of guidelines in a bullet-point structure.
\subsection{Heuristics}
% What are the suggested configurations for a technique?
% What are the guidelines for the application of a technique to a problem instance?

\begin{itemize}
	\item 
\end{itemize}



% References: Deeper understanding
% The references element description includes a listing of both primary sources of information about the technique as well as useful introductory sources for novices to gain a deeper understanding of the theory and application of the technique. The description consists of hand-selected reference material including books, peer reviewed conference papers, journal articles, and potentially websites. A bullet-pointed structure is suggested.
\subsection{References}
% What are the primary sources for a technique?
% What are the suggested reference sources for learning more about a technique?

% primary sources
\subsubsection{Primary Sources}


% more info
\subsubsection{More Information}



\putbib
\end{bibunit}


\newpage\begin{bibunit}% The Clever Algorithms Project: http://www.CleverAlgorithms.com
% (c) Copyright 2011 Jason Brownlee. Some Rights Reserved. 
% This work is licensed under a Creative Commons Attribution-Noncommercial-Share Alike 2.5 Australia License.

% Name
% The algorithm name defines the canonical name used to refer to the technique, in addition to common aliases, abbreviations, and acronyms. The name is used in terms of the heading and sub-headings of an algorithm description.
\section{Linear Discriminant Analysis} 
\label{sec:lda}
\index{Linear Discriminant Analysis}
\index{Linear Discriminant Function}
\index{Discriminant Analysis}
\index{LDA}

% other names
% What is the canonical name and common aliases for a technique?
% What are the common abbreviations and acronyms for a technique?
\emph{Linear Discriminant Analysis, Discriminant Analysis, Discriminant Function Analysis, Linear Discriminant Function, LDA.}

% Taxonomy: Lineage and locality
% The algorithm taxonomy defines where a techniques fits into the field, both the specific subfields of Computational Intelligence and Biologically Inspired Computation as well as the broader field of Artificial Intelligence. The taxonomy also provides a context for determining the relation- ships between algorithms. The taxonomy may be described in terms of a series of relationship statements or pictorially as a venn diagram or a graph with hierarchical structure.
\subsection{Taxonomy}
% To what fields of study does a technique belong?
% What are the closely related approaches to a technique?
Linear Discriminant Analysis is a classification method.

Used for classification and for dimensionality reduction.

What is Fisher Linear Discriminant Analysis?

How is it related to Quadratic Discriminant Analysis?
Factor Analysis?
PCA

Related to linear regression which provides a linear model for a numeric rather than categorical.


% Strategy: Problem solving plan
% The strategy is an abstract description of the computational model. The strategy describes the information processing actions a technique shall take in order to achieve an objective. The strategy provides a logical separation between a computational realization (procedure) and a analogous system (metaphor). A given problem solving strategy may be realized as one of a number specific algorithms or problem solving systems. The strategy description is textual using information processing and algorithmic terminology.
\subsection{Strategy}
% What is the information processing objective of a technique?
% What is a techniques plan of action?

used for classification and dimensionality reduction
kind of like regression
linear model for a nominal (categorical) dependent variable.


% help me use this technique
\subsection{Overview}

% what it is good at
\subsubsection{Features}

\begin{itemize}
	\item Used for prediction of a nominal (categorical) dependent variable, cannot be used for regression.
\end{itemize}

% what it is not good at
\subsubsection{Limitations}

\begin{itemize}
	\item 
\end{itemize}


% sample script in R
\subsection{Code Listing}

MASS package has LDA

see \url{http://www.statmethods.net/advstats/discriminant.html}

% References: Deeper understanding
% The references element description includes a listing of both primary sources of information about the technique as well as useful introductory sources for novices to gain a deeper understanding of the theory and application of the technique. The description consists of hand-selected reference material including books, peer reviewed conference papers, journal articles, and potentially websites. A bullet-pointed structure is suggested.
\subsection{References}
% What are the primary sources for a technique?
% What are the suggested reference sources for learning more about a technique?

% primary sources
\subsubsection{Primary Sources}



% more info
\subsubsection{More Information}





% END
\putbib\end{bibunit}
\newpage\begin{bibunit}% The Clever Algorithms Project: http://www.CleverAlgorithms.com
% (c) Copyright 2013 Jason Brownlee. Some Rights Reserved. 
% This work is licensed under a Creative Commons Attribution-Noncommercial-Share Alike 2.5 Australia License.


% Name
% The algorithm name defines the canonical name used to refer to the technique, in addition to common aliases, abbreviations, and acronyms. The name is used in terms of the heading and sub-headings of an algorithm description.
\section{Quadratic Discriminant Analysis} 
\label{sec:qda}
\index{Quadratic Discriminant Analysis}
\index{QDA}

% other names
% What is the canonical name and common aliases for a technique?
% What are the common abbreviations and acronyms for a technique?
\emph{Quadratic Discriminant Analysis, QDA.}

% Taxonomy: Lineage and locality
% The algorithm taxonomy defines where a techniques fits into the field, both the specific subfields of Computational Intelligence and Biologically Inspired Computation as well as the broader field of Artificial Intelligence. The taxonomy also provides a context for determining the relation- ships between algorithms. The taxonomy may be described in terms of a series of relationship statements or pictorially as a venn diagram or a graph with hierarchical structure.
\subsection{Taxonomy}
% To what fields of study does a technique belong?
% What are the closely related approaches to a technique?
Quadratic Discriminant Analysis is a classification method.


% Strategy: Problem solving plan
% The strategy is an abstract description of the computational model. The strategy describes the information processing actions a technique shall take in order to achieve an objective. The strategy provides a logical separation between a computational realization (procedure) and a analogous system (metaphor). A given problem solving strategy may be realized as one of a number specific algorithms or problem solving systems. The strategy description is textual using information processing and algorithmic terminology.
\subsection{Strategy}
% What is the information processing objective of a technique?
% What is a techniques plan of action?



% help me use this technique
\subsection{Overview}

% what it is good at
\subsubsection{Features}

\begin{itemize}
	\item 
\end{itemize}

% what it is not good at
\subsubsection{Limitations}

\begin{itemize}
	\item 
\end{itemize}


% sample script in R
\subsection{Code Listing}


MASS package


% References: Deeper understanding
% The references element description includes a listing of both primary sources of information about the technique as well as useful introductory sources for novices to gain a deeper understanding of the theory and application of the technique. The description consists of hand-selected reference material including books, peer reviewed conference papers, journal articles, and potentially websites. A bullet-pointed structure is suggested.
\subsection{References}
% What are the primary sources for a technique?
% What are the suggested reference sources for learning more about a technique?

% primary sources
\subsubsection{Primary Sources}



% more info
\subsubsection{More Information}





% END
\putbib\end{bibunit}
\newpage\begin{bibunit}% The Clever Algorithms Project: http://www.CleverAlgorithms.com
% (c) Copyright 2013 Jason Brownlee. Some Rights Reserved. 
% This work is licensed under a Creative Commons Attribution-Noncommercial-Share Alike 2.5 Australia License.


% Name
% The algorithm name defines the canonical name used to refer to the technique, in addition to common aliases, abbreviations, and acronyms. The name is used in terms of the heading and sub-headings of an algorithm description.
\section{Flexible Discriminant Analysis} 
\label{sec:fda}
\index{Flexible Discriminant Analysis}
\index{FDA}

% other names
% What is the canonical name and common aliases for a technique?
% What are the common abbreviations and acronyms for a technique?
\emph{Flexible Discriminant Analysis, FDA.}

% Taxonomy: Lineage and locality
% The algorithm taxonomy defines where a techniques fits into the field, both the specific subfields of Computational Intelligence and Biologically Inspired Computation as well as the broader field of Artificial Intelligence. The taxonomy also provides a context for determining the relation- ships between algorithms. The taxonomy may be described in terms of a series of relationship statements or pictorially as a venn diagram or a graph with hierarchical structure.
\subsection{Taxonomy}
% To what fields of study does a technique belong?
% What are the closely related approaches to a technique?
Flexible Discriminant Analysis is a classification method.


% Strategy: Problem solving plan
% The strategy is an abstract description of the computational model. The strategy describes the information processing actions a technique shall take in order to achieve an objective. The strategy provides a logical separation between a computational realization (procedure) and a analogous system (metaphor). A given problem solving strategy may be realized as one of a number specific algorithms or problem solving systems. The strategy description is textual using information processing and algorithmic terminology.
\subsection{Strategy}
% What is the information processing objective of a technique?
% What is a techniques plan of action?



% help me use this technique
\subsection{Overview}

% what it is good at
\subsubsection{Features}

\begin{itemize}
	\item 
\end{itemize}

% what it is not good at
\subsubsection{Limitations}

\begin{itemize}
	\item 
\end{itemize}


% sample script in R
\subsection{Code Listing}

mda package


% References: Deeper understanding
% The references element description includes a listing of both primary sources of information about the technique as well as useful introductory sources for novices to gain a deeper understanding of the theory and application of the technique. The description consists of hand-selected reference material including books, peer reviewed conference papers, journal articles, and potentially websites. A bullet-pointed structure is suggested.
\subsection{References}
% What are the primary sources for a technique?
% What are the suggested reference sources for learning more about a technique?

% primary sources
\subsubsection{Primary Sources}



% more info
\subsubsection{More Information}





% END
\putbib\end{bibunit}
\newpage\begin{bibunit}% The Clever Algorithms Project: http://www.CleverAlgorithms.com
% (c) Copyright 2011 Jason Brownlee. Some Rights Reserved. 
% This work is licensed under a Creative Commons Attribution-Noncommercial-Share Alike 2.5 Australia License.

% Name
% The algorithm name defines the canonical name used to refer to the technique, in addition to common aliases, abbreviations, and acronyms. The name is used in terms of the heading and sub-headings of an algorithm description.
\section{Mixture Discriminant Analysis} 
\label{sec:mda}
\index{Mixture Discriminant Analysis}
\index{MDA}

% other names
% What is the canonical name and common aliases for a technique?
% What are the common abbreviations and acronyms for a technique?
\emph{Mixture Discriminant Analysis, MDA.}

% Taxonomy: Lineage and locality
% The algorithm taxonomy defines where a techniques fits into the field, both the specific subfields of Computational Intelligence and Biologically Inspired Computation as well as the broader field of Artificial Intelligence. The taxonomy also provides a context for determining the relation- ships between algorithms. The taxonomy may be described in terms of a series of relationship statements or pictorially as a venn diagram or a graph with hierarchical structure.
\subsection{Taxonomy}
% To what fields of study does a technique belong?
% What are the closely related approaches to a technique?
Mixture Discriminant Analysis is a classification method.


% Strategy: Problem solving plan
% The strategy is an abstract description of the computational model. The strategy describes the information processing actions a technique shall take in order to achieve an objective. The strategy provides a logical separation between a computational realization (procedure) and a analogous system (metaphor). A given problem solving strategy may be realized as one of a number specific algorithms or problem solving systems. The strategy description is textual using information processing and algorithmic terminology.
\subsection{Strategy}
% What is the information processing objective of a technique?
% What is a techniques plan of action?



% help me use this technique
\subsection{Overview}

% what it is good at
\subsubsection{Features}

\begin{itemize}
	\item 
\end{itemize}

% what it is not good at
\subsubsection{Limitations}

\begin{itemize}
	\item 
\end{itemize}


% sample script in R
\subsection{Code Listing}

mda package


% References: Deeper understanding
% The references element description includes a listing of both primary sources of information about the technique as well as useful introductory sources for novices to gain a deeper understanding of the theory and application of the technique. The description consists of hand-selected reference material including books, peer reviewed conference papers, journal articles, and potentially websites. A bullet-pointed structure is suggested.
\subsection{References}
% What are the primary sources for a technique?
% What are the suggested reference sources for learning more about a technique?

% primary sources
\subsubsection{Primary Sources}



% more info
\subsubsection{More Information}





% END
\putbib\end{bibunit}

