% The Clever Algorithms Project: http://www.CleverAlgorithms.com
% (c) Copyright 2010 Jason Brownlee. Some Rights Reserved. 
% This work is licensed under a Creative Commons Attribution-Noncommercial-Share Alike 2.5 Australia License.

% This is a chapter

\renewcommand{\bibsection}{\subsection{\bibname}}
\begin{bibunit}

\chapter{Ensembles}
\label{ch:ensembles}
\index{Ensembles}

\section{Overview}
This chapter describes Ensembles methods.


% Strategy: Problem solving plan
% The strategy is an abstract description of the computational model. The strategy describes the information processing actions a technique shall take in order to achieve an objective. The strategy provides a logical separation between a computational realization (procedure) and a analogous system (metaphor). A given problem solving strategy may be realized as one of a number specific algorithms or problem solving systems. The strategy description is textual using information processing and algorithmic terminology.
\subsection{Strategy}
% What is the information processing objective of a technique?
% What is a techniques plan of action?


efficient - weak learners are easier to train than finding the best model
difficulty in selecting a model
Weak learner and basis function are synonyms.
There is now theory on why ensembles work (performing a Monte Carlo simulation for an integration problem).
Best results from combining low bias, high-variance learners

strategies
- vary the models (functions)
- vary the data

compmenent of model selection - combines measures of fit in cross validation
ensembles combine fitted values (predictions)


BOOSTING
prepare each tree on a weighted sample
fix the mistakes made before
more clever version of bagging
build tree, assign weights based on mistakes, build next tree, so on
weighted averages of observations when splitting

gradient descent of a loss function... modern understanding - gradient boosting

control the size of the trees - quite small - faster

train beyond perfect train data - test error gets better!
makes the problem harder and harder - but the error does start increasing again

MODERN:
building an expanding basis function - stepwise additive modelling - does it iteratively - forward stage wise learning
like basis pursuit is of the same form - matching pursuit




% Heuristics: Usage guidelines
% The heuristics element describe the commonsense, best practice, and demonstrated rules for applying and configuring a parameterized algorithm. The heuristics relate to the technical details of the techniques procedure and data structures for general classes of application (neither specific implementations not specific problem instances). The heuristics are described textually, such as a series of guidelines in a bullet-point structure.
\subsection{Heuristics}
% What are the suggested configurations for a technique?
% What are the guidelines for the application of a technique to a problem instance?


\begin{itemize}
	\item Many weak learners can out-perform a single strong learner.
	\item The diversity of the ensemble set is important property of success. Don't want all models to tell the same story - want different points of view.
	\item Model must be better than random, models must be independent (data, modeled function, etc).
\end{itemize}


% References: Deeper understanding
% The references element description includes a listing of both primary sources of information about the technique as well as useful introductory sources for novices to gain a deeper understanding of the theory and application of the technique. The description consists of hand-selected reference material including books, peer reviewed conference papers, journal articles, and potentially websites. A bullet-pointed structure is suggested.
\subsection{References}
% What are the primary sources for a technique?
% What are the suggested reference sources for learning more about a technique?

% primary sources
\subsubsection{Primary Sources}



% more info
\subsubsection{More Information}

Early overview that was cited a lot \cite{Dietterich2000}.

\putbib
\end{bibunit}


\newpage\begin{bibunit}% The Clever Algorithms Project: http://www.CleverAlgorithms.com
% (c) Copyright 2011 Jason Brownlee. Some Rights Reserved. 
% This work is licensed under a Creative Commons Attribution-Noncommercial-Share Alike 2.5 Australia License.

% Name
% The algorithm name defines the canonical name used to refer to the technique, in addition to common aliases, abbreviations, and acronyms. The name is used in terms of the heading and sub-headings of an algorithm description.
\section{Bootstrapped Aggregation} 
\label{sec:bagging}
\index{Bagging}
\index{Bootstrapped Aggregation}

% other names
% What is the canonical name and common aliases for a technique?
% What are the common abbreviations and acronyms for a technique?
\emph{Bagging, Bootstrapped Aggregation}

% Taxonomy: Lineage and locality
% The algorithm taxonomy defines where a techniques fits into the field, both the specific subfields of Computational Intelligence and Biologically Inspired Computation as well as the broader field of Artificial Intelligence. The taxonomy also provides a context for determining the relation- ships between algorithms. The taxonomy may be described in terms of a series of relationship statements or pictorially as a venn diagram or a graph with hierarchical structure.
\subsection{Taxonomy}
% To what fields of study does a technique belong?
% What are the closely related approaches to a technique?
Bootstrapped Aggregation (Bagging) is an Ensemble algorithm.
Bagging is a special case of Stagewise Additive Modeling - General Forward Stagewise Additive Modeling


% Strategy: Problem solving plan
% The strategy is an abstract description of the computational model. The strategy describes the information processing actions a technique shall take in order to achieve an objective. The strategy provides a logical separation between a computational realization (procedure) and a analogous system (metaphor). A given problem solving strategy may be realized as one of a number specific algorithms or problem solving systems. The strategy description is textual using information processing and algorithmic terminology.
\subsection{Strategy}
% What is the information processing objective of a technique?
% What is a techniques plan of action

a variance reduction technique
 each partition is an iid

partition data into random subsets, with replacement - called bootstrap replicates
train a model on each partition
average the results

not suited to all models - not good for svm
svm variance is not high

% help me use this technique
\subsection{Overview}

% what it is good at
\subsubsection{Features}

\begin{itemize}
	\item 
\end{itemize}

% what it is not good at
\subsubsection{Limitations}

\begin{itemize}
	\item 
\end{itemize}


% sample script in R
\subsection{Code Listing}
% listing
Listing~\ref{ipred_bagging} provides a code listing of the Bagging algorithm in R.
% algorithm and package
The example uses the \texttt{bagging()} function in the \texttt{ipred} package \cite{Peters2011} that provides bootstrap aggregation with trees via the \texttt{rpart()} of the \texttt{rpart} package.
% TODO more about the package and the functions capabilities
Peters, et al. also provide a vignette of the \texttt{ipred} package with more examples and references \cite{Peters2011a}.

% problem
The test problem is a three-dimensional dataset, where the $x$, $y$ and $z$ attributes are numerical and drawn from normal distributions with different means. The objective is to predict the categorical dependent $z$ given the independents $x$ and $y$.

% code listing
\lstinputlisting[firstline=7,language=r,caption={Example of the Bagging algorithm in R using the \texttt{bagging()} function of the \texttt{ipred} package.}, label=ipred_bagging]{../src/algorithms/ensemble/ipred_bagging.R}

% other packages
Other packages provide an implementation of the bagging algorithm including the \texttt{adabag} package \cite{Cortes2011}, the \texttt{ada} package that only provides binary classification, and the \texttt{caret} package provides a framework for bagging \cite{Kuhn2011}.

% References: Deeper understanding
% The references element description includes a listing of both primary sources of information about the technique as well as useful introductory sources for novices to gain a deeper understanding of the theory and application of the technique. The description consists of hand-selected reference material including books, peer reviewed conference papers, journal articles, and potentially websites. A bullet-pointed structure is suggested.
\subsection{References}
% What are the primary sources for a technique?
% What are the suggested reference sources for learning more about a technique?

% primary sources
\subsubsection{Primary Sources}

Developed by Breiman \cite{Breiman1996}

% more info
\subsubsection{More Information}



% END
\putbib\end{bibunit}
\newpage\begin{bibunit}% The Clever Algorithms Project: http://www.CleverAlgorithms.com
% (c) Copyright 2011 Jason Brownlee. Some Rights Reserved. 
% This work is licensed under a Creative Commons Attribution-Noncommercial-Share Alike 2.5 Australia License.

% Name
% The algorithm name defines the canonical name used to refer to the technique, in addition to common aliases, abbreviations, and acronyms. The name is used in terms of the heading and sub-headings of an algorithm description.
\section{AdaBoost} 
\label{sec:adaboost}
\index{AdaBoost}

% other names
% What is the canonical name and common aliases for a technique?
% What are the common abbreviations and acronyms for a technique?
\emph{AdaBoost}

% Taxonomy: Lineage and locality
% The algorithm taxonomy defines where a techniques fits into the field, both the specific subfields of Computational Intelligence and Biologically Inspired Computation as well as the broader field of Artificial Intelligence. The taxonomy also provides a context for determining the relation- ships between algorithms. The taxonomy may be described in terms of a series of relationship statements or pictorially as a venn diagram or a graph with hierarchical structure.
\subsection{Taxonomy}
% To what fields of study does a technique belong?
% What are the closely related approaches to a technique?
AdaBoost is an Ensemble algorithm.

% Strategy: Problem solving plan
% The strategy is an abstract description of the computational model. The strategy describes the information processing actions a technique shall take in order to achieve an objective. The strategy provides a logical separation between a computational realization (procedure) and a analogous system (metaphor). A given problem solving strategy may be realized as one of a number specific algorithms or problem solving systems. The strategy description is textual using information processing and algorithmic terminology.
\subsection{Strategy}
% What is the information processing objective of a technique?
% What is a techniques plan of action?
A textual description of the information processing strategy.

% Heuristics: Usage guidelines
% The heuristics element describe the commonsense, best practice, and demonstrated rules for applying and configuring a parameterized algorithm. The heuristics relate to the technical details of the techniques procedure and data structures for general classes of application (neither specific implementations not specific problem instances). The heuristics are described textually, such as a series of guidelines in a bullet-point structure.
\subsection{Heuristics}
% What are the suggested configurations for a technique?
% What are the guidelines for the application of a technique to a problem instance?

\begin{itemize}
	\item todo
\end{itemize}

% References: Deeper understanding
% The references element description includes a listing of both primary sources of information about the technique as well as useful introductory sources for novices to gain a deeper understanding of the theory and application of the technique. The description consists of hand-selected reference material including books, peer reviewed conference papers, journal articles, and potentially websites. A bullet-pointed structure is suggested.
\subsection{References}
% What are the primary sources for a technique?
% What are the suggested reference sources for learning more about a technique?

% primary sources
\subsubsection{Primary Sources}

Proposed by Freund and Schapire \cite{Freund1997}.

% more info
\subsubsection{More Information}



% END
\putbib\end{bibunit}
\newpage\begin{bibunit}% The Clever Algorithms Project: http://www.CleverAlgorithms.com
% (c) Copyright 2011 Jason Brownlee. Some Rights Reserved. 
% This work is licensed under a Creative Commons Attribution-Noncommercial-Share Alike 2.5 Australia License.

% Name
% The algorithm name defines the canonical name used to refer to the technique, in addition to common aliases, abbreviations, and acronyms. The name is used in terms of the heading and sub-headings of an algorithm description.
\section{Gradient Boosting} 
\label{sec:gradientboosting}
\index{Gradient Boosting}

% other names
% What is the canonical name and common aliases for a technique?
% What are the common abbreviations and acronyms for a technique?
\emph{Gradient Boosting, Gradient Boosting Machine, GBM}

% Taxonomy: Lineage and locality
% The algorithm taxonomy defines where a techniques fits into the field, both the specific subfields of Computational Intelligence and Biologically Inspired Computation as well as the broader field of Artificial Intelligence. The taxonomy also provides a context for determining the relation- ships between algorithms. The taxonomy may be described in terms of a series of relationship statements or pictorially as a venn diagram or a graph with hierarchical structure.
\subsection{Taxonomy}
% To what fields of study does a technique belong?
% What are the closely related approaches to a technique?
Gradient Boosting is an Ensemble algorithm.

Gradient Boosted Decision Trees (GBDT)
Gradient Boosted Regression Trees (GBRT) - application of the method to regression trees
Stochastic Gradient Boosting
Gradient Boost
Gradient Tree Boosting
TreeBoost
Generalized Boosting Model
Functional Gradient Boosting
Gradient Boosted Models
Multiple Additive Regression Trees (MART) - commercial
TreeNet - commercial


% Strategy: Problem solving plan
% The strategy is an abstract description of the computational model. The strategy describes the information processing actions a technique shall take in order to achieve an objective. The strategy provides a logical separation between a computational realization (procedure) and a analogous system (metaphor). A given problem solving strategy may be realized as one of a number specific algorithms or problem solving systems. The strategy description is textual using information processing and algorithmic terminology.
\subsection{Strategy}
% What is the information processing objective of a technique?
% What is a techniques plan of action?


for regression only?



% help me use this technique
\subsection{Overview}

% what it is good at
\subsubsection{Features}

\begin{itemize}
	\item 
\end{itemize}

% what it is not good at
\subsubsection{Limitations}

\begin{itemize}
	\item 
\end{itemize}


% sample script in R
\subsection{Code Listing}
% listing
Listing~\ref{gbm_gradient_boosting} provides a code listing of the Gradient Boosting algorithm in R.
% algorithm and package
The example uses the \texttt{gbm()} function in the \texttt{gbm} package \cite{Ridgeway2007a}. The \texttt{gbm()} function makes use of the \texttt{rpart()} function from the \texttt{rpart} package to create an ensemble of trees as the weak learners.
% TODO more about the package and the functions capabilities
Ridgeway also provides a vignette of the \texttt{gbm} package with more examples and references \cite{Ridgeway2007}.

% problem
The test problem is a three-dimensional dataset, where the $x$, $y$ and $z$ attributes are numerical and drawn from normal distributions with different means. The objective is to predict the categorical dependent $z$ given the independents $x$ and $y$.

% code listing
\lstinputlisting[firstline=7,language=r,caption={Example of the Gradient Boosting algorithm in R using the \texttt{gbm()} function of the \texttt{gbm} package.}, label=gbm_gradient_boosting]{../src/algorithms/ensemble/gbm_gradient_boosting.R}

% other packages


% References: Deeper understanding
% The references element description includes a listing of both primary sources of information about the technique as well as useful introductory sources for novices to gain a deeper understanding of the theory and application of the technique. The description consists of hand-selected reference material including books, peer reviewed conference papers, journal articles, and potentially websites. A bullet-pointed structure is suggested.
\subsection{References}
% What are the primary sources for a technique?
% What are the suggested reference sources for learning more about a technique?

% primary sources
\subsubsection{Primary Sources}

Gradient Boosting was proposed by Friedman \cite{Friedman2001}.

% more info
\subsubsection{More Information}



% END
\putbib\end{bibunit}
\newpage\begin{bibunit}% The Clever Algorithms Project: http://www.CleverAlgorithms.com
% (c) Copyright 2011 Jason Brownlee. Some Rights Reserved. 
% This work is licensed under a Creative Commons Attribution-Noncommercial-Share Alike 2.5 Australia License.

% Name
% The algorithm name defines the canonical name used to refer to the technique, in addition to common aliases, abbreviations, and acronyms. The name is used in terms of the heading and sub-headings of an algorithm description.
\section{Random Forest} 
\label{sec:randomforest}
\index{Random Forest}

% other names
% What is the canonical name and common aliases for a technique?
% What are the common abbreviations and acronyms for a technique?
\emph{Random Forest, RF}

% Taxonomy: Lineage and locality
% The algorithm taxonomy defines where a techniques fits into the field, both the specific subfields of Computational Intelligence and Biologically Inspired Computation as well as the broader field of Artificial Intelligence. The taxonomy also provides a context for determining the relation- ships between algorithms. The taxonomy may be described in terms of a series of relationship statements or pictorially as a venn diagram or a graph with hierarchical structure.
\subsection{Taxonomy}
% To what fields of study does a technique belong?
% What are the closely related approaches to a technique?
Random Forest is an Ensemble algorithm. 

Random Forest is a registered trademark of Leo Breiman and Adele Cutler as are many variations.

% Strategy: Problem solving plan
% The strategy is an abstract description of the computational model. The strategy describes the information processing actions a technique shall take in order to achieve an objective. The strategy provides a logical separation between a computational realization (procedure) and a analogous system (metaphor). A given problem solving strategy may be realized as one of a number specific algorithms or problem solving systems. The strategy description is textual using information processing and algorithmic terminology.
\subsection{Strategy}
% What is the information processing objective of a technique?
% What is a techniques plan of action?

fancy version of bagging
shakes up the data more
pick a random subset of features and use them as candidates for splitting
works much better than original bagging
random subset - de-correlates the trees



% help me use this technique
\subsection{Overview}

% what it is good at
\subsubsection{Features}

\begin{itemize}
	\item 
\end{itemize}

% what it is not good at
\subsubsection{Limitations}

\begin{itemize}
	\item 
\end{itemize}


% sample script in R
\subsection{Code Listing}
% listing
Listing~\ref{randomforest_random_forest} provides a code listing of the Random Forest algorithm in R.
% algorithm and package
The example uses the \texttt{randomForest()} function in the \texttt{randomForest} package.
% TODO more about the package and the functions capabilities
Liaw, et al. provide a beginners introduction to \texttt{randomForest} package \cite{Liaw2002}.

% problem
The test problem is a three-dimensional dataset, where the $x$, $y$ and $z$ attributes are numerical and drawn from normal distributions with different means. 

% TODO use a sample problem with textual labels

% code listing
\lstinputlisting[firstline=7,language=r,caption={Example of the Random Forest algorithm in R using the \texttt{randomForest()} function of the \texttt{randomForest} package.}, label=randomforest_random_forest]{../src/algorithms/ensemble/randomforest_random_forest.R}

% other packages
Other packages provide implementations of the Random Forest algorithm, such as the \texttt{party} pacakge. Strobl, et al. provide introduction to \texttt{party} pacakge and heuristics on what Random Forest package to use for given problem \cite{Strobl2009} .


% References: Deeper understanding
% The references element description includes a listing of both primary sources of information about the technique as well as useful introductory sources for novices to gain a deeper understanding of the theory and application of the technique. The description consists of hand-selected reference material including books, peer reviewed conference papers, journal articles, and potentially websites. A bullet-pointed structure is suggested.
\subsection{References}
% What are the primary sources for a technique?
% What are the suggested reference sources for learning more about a technique?

% primary sources
\subsubsection{Primary Sources}

Developed by Breiman \cite{Breiman2001}.

Practical usage information in \cite{Breiman2003}

% more info
\subsubsection{More Information}



% END
\putbib\end{bibunit}

