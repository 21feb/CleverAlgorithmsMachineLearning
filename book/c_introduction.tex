% The Clever Algorithms Project: http://www.CleverAlgorithms.com
% (c) Copyright 2010 Jason Brownlee. Some Rights Reserved. 
% This work is licensed under a Creative Commons Attribution-Noncommercial-Share Alike 2.5 Australia License.

% This is a chapter

% Argument and background information the user requires to read and understand the book
\chapter{Introduction}
\label{chap:intro}
% welcome
\emph{Welcome to Clever Algorithms!} This is a handbook of recipes for computational problem solving techniques from the field of Statistical Machine Learning. 
% they are for using
Clever Algorithms are interesting, practical, and fun to learn about and implement.
% briefly the audience
Research scientists may be interested in browsing algorithm in search of an interesting system to investigate. Developers and software engineers may compare various problem solving algorithms and technique-specific guidelines. Practitioners, students, and interested amateurs may implement state-of-the-art algorithms to address business or scientific needs, or simply play with the fascinating systems they represent.

% briefly book overview
This introductory chapter provides relevant background information on Machine Learning and Algorithms. The core of the book provides a large corpus of algorithms presented in a complete and consistent manner. The final chapter covers some advanced topics to consider once a number of algorithms have been mastered. This book has been designed as a reference text, where specific techniques are looked up, or where the algorithms across whole fields of study can be browsed, rather than being read cover-to-cover. This book is an algorithm handbook and a technique guidebook, and I hope you find something useful.

% sections
\newpage\begin{bibunit}% The Clever Algorithms Project: http://www.CleverAlgorithms.com
% (c) Copyright 2013 Jason Brownlee. Some Rights Reserved. 
% This work is licensed under a Creative Commons Attribution-Noncommercial-Share Alike 2.5 Australia License.


% Machine Learning
\section{Machine Learning} 
\label{intro:machinelearning}
\index{Machine Learning}

% my def
Machine learning is concerned with methods that process empirical data to construct models that contain the salient or underlying probability distribution in the data. Such models may then be applied to address problems or make decisions regarding new data or data unseen by the system.
% this section
This section provides some standard definitions of machine learning and goes on to summarizes some of the over arching fields and related sub-fields.

% official definitions
\subsection{Definitions}
\label{subsec:definitions}
\index{Machine Learning!Definition}
This section visits standard definitions and descriptions of the field from some of the classical texts on statistical pattern recognition and machine learning. 

Fukunaga's classic text \emph{Introduction to Statistical Pattern Recognition} summarizes the field (\cite{Fukunaga1990}, page 2--3):

\begin{quotation}
... pattern recognition, or decision-making in a broader sense, may be considered as a problem of estimating density functions in a high-dimensional space and dividing the space into the regions of categories or classes. Because of this view, mathematical statistics forms the foundation of the subject.
\end{quotation} 

Hastie et~al.\ seminal text \emph{The Elements of Statistical Learning: Data Mining, Inference, and Prediction} describes a typical scenario (\cite{Hastie2009}, pages 1--2):

\begin{quotation}
... we have an outcome measurement, usually quantitative ... or categorical ..., that we wish to predict based on a set of features ... We have a training set of data, in which we observe the outcome and feature measurements for a set of objects ... Using this data we build a prediction model, or learner, which will enable us to predict the outcome for new unseen objects. A good learner is one that accurately predicts such an outcome.
\end{quotation} 

Duda et~al.\ classic text \emph{Pattern Classification} suggests (\cite{Duda2001}, page 16):

\begin{quotation}
In the broadest sense, any method that incorporates information from training samples in the design of a classifier employs learning. ... Learning refers to some form of algorithm for reducing the error on a set of training data.
\end{quotation}

Mitchell's classic text \emph{Machine Learning} suggests (\cite{Mitchell1997}, page xv):

\begin{quotation}
The field of machine learning is concerned with the question of how to construct computer programs that automatically improve with their experience.
\end{quotation}

Mitchell also provides a precise definition which is popular in the field (\cite{Mitchell1997}, page 2):

\begin{quotation}
A computer program is said to learn from experience E with respect to some class of tasks T and performance measure P, if its performance at tasks in T, as measure by P, improves with experience E.
\end{quotation}

Mitchell suggests that the field of Machine Learning seeks to answer the question \cite{Mitchell2006}

\begin{quotation}
How can we build computer systems that automatically improve with experience, and what are the fundamental laws that govern all learning processes?
\end{quotation}

% Related fields
\subsection{Related Fields}
\label{subsec:related_fields}
\index{Machine Learning!Related Fields}
Machine Learning is a cross-disciplinary field that draws upon many related and overarching fields. This section describes some of those fields related to the study of Machine Learning in order to provide context.

% Artificial Intelligence
\subsubsection{Artificial Intelligence}
\index{Artificial Intelligence}
The field of Artificial Intelligence (AI) coalesced in the 1950s drawing on an understanding of the brain from neuroscience, the new mathematics of information theory, control theory referred to as cybernetics, and the dawn of the digital computer. AI is a cross-disciplinary field of research that is generally concerned with developing and investigating systems that operate or act intelligently. It is considered a discipline in the field of computer science given the strong focus on computation. The study of artificial intelligence is concerned with investigating mechanisms that underlie intelligence and intelligence behavior.

Russell and Norvig provide a perspective that defines artificial intelligence in four categories: 1) systems that think like humans, 2) systems that act like humans, 3) systems that think rationally, 4) systems that act rationally \cite{Russell2009}. In their definition, acting like a human suggests that a system can do some specific things humans can do, this includes fields such as the Turing test, natural language processing, automated reasoning, knowledge representation, machine learning, computer vision, and robotics. Thinking like a human suggests systems that model the cognitive information processing properties of humans, for example a general problem solver and systems that build internal models of their world. Thinking rationally suggests laws of rationalism and structured thought, such as syllogisms and formal logic. Finally, acting rationally suggests systems that do rational things such as expected utility maximization and rational agents. 

Machine Learning is concerned with developing systems that learn from experience, feature that may be considered a signature of intelligence. As such, machine learning may be considered a sub-field of artificial intelligence.

% Data Mining
\subsubsection{Data Mining}
\index{Data Mining}
Broadly, the field of Data Mining may be considered the practical application of statistical and machine learning methods to real-world problem domains. 
Witten and Frank provide a clear definition of the field (\cite{Witten2011}, page 5)

\begin{quotation}
Data mining is about solving problems by analyzing data already present in databases. ... Data mining is defined as the process of discovering patterns in data. The process must be automatic or (more usually) semiautomatic. The patterns discovered must be meaningful in that they lead to some advantage usually an economic advantage. The data is invariably present in substantial quantities.
\end{quotation}

The field traditionally uses the phrase `Knowledge Discovery in Databases' (KDD) to describe itself, focusing on the process of applying a suite of machine learning methods to solve a problem \cite{Frawley1992}. An early definition of this process proposed the following elements: Selection, Preprocessing, Transformation, Data Mining, Interpretation and Evaluation, where Data Mining is a single step in the procedure \cite{Fayyad1996a}.
% relation
This suggests Machine Learning methods as the tools that may be used in a KDD process and/or during Data Mining.

% Other
\subsubsection{Fundamentals}
This section describes those fundamental areas from which Machine Learning is rooted. 

\begin{description}
\index{Statistics}
	\item[Statistics]: The field of statistics is a branch of mathematics concerned the collection, organization, and interpretation of quantitative data. A statistical understanding of data provides a basis for machine learning methods that generalize from quantitative data, commonly characterizing the probabilistic features that underlie the observations.
	
\index{Computational Learning Theory}
	\item[Computational Learning Theory]: The field of Computational Learning Theory broadly involves the analysis of machine learning systems. The field provides the theoretical basis for types of learning and their limitations. Machine Learning methods can be developed to investigate a theoretical finding, or be made more efficient through theoretical discoveries. For more information, refer to Kearns and Vazirani \cite{Kearns1994}.
	
\index{Probability Theory}
	\item[Probability Theory]: The field of probability theory is a branch of mathematics concerned with the likelihood of random events and characterizing the distribution of occurrences. The findings and methods from probability theory provides a basis to the field of statistics. 
	
\index{Decision Theory}
	\item[Decision Theory]: The field of decision theory is concerned with rational and optimal decision making in the presence of uncertainty. The field is also closely related to game theory and probability theory. The findings from decision theory relate to the application of models in machine learning for the decisions made in discrimination and related tasks.
	
\index{Information Theory}
	\item[Information Theory]: The field of information theory from applied mathematics and computer science is concerned with the quantification of information in data. The field is based on probability theory and statistics, and focuses on the amount or change in information entropy which is the measure of information content associated with a message. Information theory is important in machine learning in compression and generalization that occurs in the preparation of a model from data samples. For more information refer to MacKay \cite{MacKay2003}.
\end{description}

The application of statistical and machine learning methods to specific domains are typically referred to as a standalone field, such as machine vision, speech recognition, machine translation, and information filtering.

%
% Taxonomies
%
\subsection{Taxonomies}
\label{sec:taxonomies}
\index{Machine Learning!Taxonomy}
Machine Learning is a large and complex multidisciplinary field. As such there are many perspectives or ways of breaking up the methods and algorithms it contains. This section considers a number of common taxonomies of Machine Learning methods.

% taxonomy by learning
We first consider a partitioning of the field by the type of learning performed by an algorithm used to prepare a model. Learning is performed from examples provided in the form of observations or samples from the domain they may be available or may have to be collected by a software agent. This data is referred to as the `training set'. 

The type of learning performed by an approach can be defined in terms of the feedback provided by the environment based on the decisions or actions taken by a model prepared from collected data.

\begin{description}
	\index{Unsupervised Learning}
	\item[Unsupervised Learning]: No feedback is provided from the environment. Methods are left to deduce structure and patterns. The most common models used in unsupervised learning include clustering algorithms that suggest natural groupings within the data.
	
	\index{Supervised Learning}
	\item[Supervised Learning]: Input data is associated with an output value or values, a relationship which can be learned and frequently assessed. This assessment may result in a performance or error measure that is used to further refine the model. A common example of problems for supervised learning include regression and classification where an ordinal or categorical output is associated with a collection of input data attributes. A model is prepared with a training dataset, then assessed with the same or a different test dataset, the results of which are fed back to the model for correction.
	
	\index{Semi-supervised Learning}
	\item[Semi-supervised Learning]: Input data may be associated with an output value or values, although only some labeled examples are provided, and/or examples may be mislabeled. This provides an intermediate that must exploit the capabilities of unsupervised learning where natural clusters are sought in the data, and supervised learning where known relationships are demonstrated between input and output data attributes. For more information see Zhu's literature survey \cite{Zhu2008} or book \cite{Zhu2009}, or the edited volume by Chapelle et~al.\ \cite{Chapelle2010}.
	
	\index{Reinforcement Learning}
	\item[Reinforcement Learning]: A model or agent learns in an environment through (likely sporadic and infrequent) reinforcements, such as rewards and/or punishments. A popular analogy is that the environment provides a critic (rather than a teacher) which makes suggestions of correctness or incorrectness without indication of why or what the correct answer might be. For more information, refer to Sutton and Bardo \cite{Sutton1998}.
\end{description}

An important distinction between machine learning methods is the relationship between the data and the nature of the model prepared from the data.

\begin{description}
	\index{Inductive Learning}
	\item[Inductive Learning]: The most common form of learning where specific examples in the form of observations from the domain are used as the basis for making generalizations to solve problems, such as natural groups or class discrimination boundaries. Generalization is based on statistics applied to the empirical observations and typically requires a large number of examples that are representative of the underlying structure (mass or density functions) in the data.
	
	\index{Analytical Learning}
	\item[Analytical Learning]: A form of learning where prior knowledge and deductive reasoning is used in conjunction with available data. Prior knowledge is used as an explanation for the features in training data, allowing generalization through logical rather than statistical reasoning. An example of analytical learning is Explanation-based Learning used in domains such as scheduling.
	
\end{description}

The process of model construction and generalization depends on the availability of data. Observations from the domain may be collected and used to prepare a model or a model may be updated as observations become available. The following summarizes the relationship between model learning and data availability.

\begin{description}
	\index{Online Learning}
	\item[Online Learning]: A system makes predictions on an instance-by-instance basis, where the actual outcome or result for the instance is provided directly or soon after the prediction is made. Once the actual outcome is known, the system can make any adjustments necessary to improve its predictive ability. The ongoing learning of an online system provides a capability to adapt to changes in data over extended periods of time.
	
	\index{Offline Learning} 
	\item[Offline Learning]: A sample of observations are collected beforehand from which the system may process for as long as needed. This sample or training dataset may be evaluated in order to derive holistic generalizations. A system maybe trained on the whole dataset or on selected sub-samples of the dataset in batch, giving the alternative name of batch learning. After the training phase has occurred, the approximation prepared by the system typically does not change for the life of the system.

\end{description}



\putbib\end{bibunit}
\newpage\begin{bibunit}% The Clever Algorithms Project: http://www.CleverAlgorithms.com
% (c) Copyright 2011 Jason Brownlee. Some Rights Reserved. 
% This work is licensed under a Creative Commons Attribution-Noncommercial-Share Alike 2.5 Australia License.

% Problem Domains
\section{Problem Domains} 
\label{intro:problemdomains}
\index{Problem Domains}

% taxonomy by problem
\subsection{Problems}
\label{subsec:problems}
There are many subtly different classes of problem's that may be addressed via machine learning methods, all generally stemming from the abstract problem of Function Approximation. Function approximation is the problem of finding a function ($f$) that approximates a target function ($g$), where typically the approximated function is selected based on a sample of observations ($x$, also referred to as the training set) taken from the unknown target function.

In Machine Learning, the function approximation formalism is used to describe general problem types commonly referred to as pattern recognition, such as classification, clustering, and curve fitting (called a decision or discrimination function). Such general problem types are described in terms of approximating an unknown Probability Density Function, which underlies the relationships in the problem space, and is represented in the sample data. This perspective of such problems is commonly referred to as statistical machine learning and/or density estimation.

% general process
The general process focuses on 1) the collection and preparation of the observations from the target function, 2) the selection and/or preparation of a model of the target function, and 3) the application and ongoing refinement of the prepared model. 
% optimization
The field of Function Optimization is related to function approximation, as many-sub-problems of function approximation may be defined as optimization problems. Many of the technique paradigms used for function approximation are differentiated based on the representation and the optimization process used to minimize error or maximize effectiveness on a given approximation problem.

% problems
The difficulty of function approximation problems center around 1) the nature of the unknown relationships between attributes and features, 2) the number (dimensionality) of attributes and features, and 3) general concerns of noise in such relationships and the dynamic availability of samples from the target function.
% other problems
Additional difficulties include the incorporation of prior knowledge (such as imbalance in samples, incomplete information and the variable reliability of data), and problems of invariant features (such as transformation, translation, rotation, scaling, and skewing of features).

The following describes some of the general sub-problems of function approximation addressed via Machine Learning methods:

\begin{description}
	\item[Feature Selection]: A feature is considered an aggregation of one-or-more attributes, where only those features that have meaning in the context of the target function are necessary to the modeling function. Feature selection methods may be used to reduce the dimensionality of a dataset, before addressing a follow-up problem such as classification. It is common to assess the information content of attributes in feature selection methods, or more crudely to determine the effect on models with and without specific attributes to determine their utility.
	
	\item[Classification]: Observations are inherently organized into labeled groups (classes) and a supervised process models an underlying discrimination function to classify unobserved samples. A system must deduce the boundaries between classes and be able to discriminate based on the class boundaries. Classification problems are addressed through a supervised learning method given labeled data from the domain is available to teach the model the class boundaries. 
	
	\item[Clustering]: Observations may be organized into natural groups based on underlying structure or features in the data. The groups are unlabeled requiring a process to model an underlying discrimination function without corrective feedback. As such, unsupervised methods are commonly used for clustering problems. Resulting data clusters may be labeled after the fact and used as the basis for follow-on classification problems.
	
	\item[Regression]: A model is prepared that provides a `best-fit' for a set of observations that may be used for interpolation over known observations and extrapolation for observations outside what has been modeled. The best-fit may be two-dimensional (line-fitting) or higher-dimensionality (surface fitting). It is common for regression problems to have a time axis.
	
	\item[Association]: Rules may be deduced between attributes in the data and may discover interesting and useful statistical patterns. Such rules may or may not be related to the predicted attributes, such as the class in a classification problem, and may useful for identifying ``important'' attributes such as in feature selection.
		
\end{description}
\putbib\end{bibunit}
\newpage\begin{bibunit}% The Clever Algorithms Project: http://www.CleverAlgorithms.com
% (c) Copyright 2011 Jason Brownlee. Some Rights Reserved. 
% This work is licensed under a Creative Commons Attribution-Noncommercial-Share Alike 2.5 Australia License.

% Considerations
\section{Considerations} 
\label{intro:considerations}
% introduction
This section provides a brief introduction into some important considerations in Machine Learning that should motivate a deeper understanding into the use of algorithm.

% Nomenclature
\subsection{Nomenclature}
\index{Machine Learning!Nomenclature}
There are many terms thrown around in the field of Machine Learning, some borrowed from Statistics, some from Artificial Intelligence, and some from Probability theory (as well as an assortment of other fields). Unless you have a grounding in many fields, the nomenclature can be confusing. This section provides a terse overview of some of the terms used to refer to data and models

Data is the starting point. Processed data is \emph{information} and information that has meaning is \emph{knowledge}. For example, an objective of Data Mining is acquiring knowledge from data using Machine Learning algorithms. We organize our data into rows and columns. A row of data may be considered an \emph{instance}, \emph{example}, \emph{event} or an \emph{observation}. The columns of our data may be referred to as \emph{attributes}, \emph{variables} or \emph{features}. 

When we are asking a question of our data we build a model. In statistics, we refer to the relationships between variables as \emph{dependent} and those variables without \emph{relationships} as independent. In the context of a model, we would say that we wish to predict a dependent variable based on one or more independent variables. Mathematically, a model encapsulates the relationships between the dependent and independent variables as a \emph{target function}. We may conceptualize an idealized version of this function that the model seeks to approximate or \emph{fit} over a training process. 

When we are looking to train or fit a model, we take a sample of our data and call it the \emph{training dataset} and a separate independent sample called the \emph{testing dataset} for assessing the unbiased performance of the resulting model. A model should `learn' from the training set and generalize this knowledge to the test dataset and beyond. A model that is not trained enough and does not have enough knowledge is \emph{under-fit}, whereas a model that is trained too much cannot generalize (is too specialized on the training dataset) and is said to be \emph{over-fit}.

% Bias-Variance
\subsection{Bias-Variance Trade-off}
\index{Bias-Variance Trade-off}
A central theme in Machine Learning is walking the trade-off between the \emph{bias} and \emph{variance} in a given model for a problem dataset.

\begin{itemize}
	\item \textbf{Bias}: Best understood as the assumptions that constrain a model. Classically, this was referred to as Inductive Bias in Machine Learning. 
	\item \textbf{Variance}: Best understood as the sensitivity of the model to the specific data on which it is trained.
\end{itemize}

Bias and Variance have a dichotic relationship where an increase in bias results in a decrease in variance for a given model, and vice versa. 
Practically, one may consider each problem to have sweet-spot in this trade-off, a minimum bias and a minimum variance. Typically, the objective of a problem may be to get as close as possible to this sweet-spot with a given model (but this may not always be the case). 

\index{Learning Curve}
This trade-off is a useful motivation in fitting a model for a given dataset. It can be used as a lens through which summary statistics can be understood. An good example is that of a learning curve which is a plot of a models error score during training (such as the Sum Squared Residual error during a least squares optimization in linear regression). One can plot a learning curve for a training dataset and a test dataset during training. One may see the error curves decrease on the training and test set initially, although over an extended training period, the training error may continue to decrease and the test error may start to rise again.

This is an example of model over-fitting where the model is tuned for performance the training data at the expense of performance on the test dataset, the model has a high variance. The inflection point in the learning curve graph where before the test set error starts to rise again may be an example of the sweet-spot in the bias-variance trade-off.

Dietterich and Kong provide an interesting discussion of this trade-off in the context of decision trees \cite{Dietterich1995}. In their paper they present a more nuanced geometrical metaphor for model bias:

\begin{itemize}
	\item \textbf{Absolute Bias}: This is the general area in the search space of possible target functions available. The absolute bias is the acknowledgment that a selected method always reduces the set of target functions to a general sub-set.
	\item \textbf{Relative Bias}: This is the preference within the general area of the search space of possible target functions. The relative bias is the acknowledgement that a target function is more likely to come from one set of functions than another.
\end{itemize} 

This understanding of bias highlights the fact that some bias must be adopted in the selection of a model, and that there is a need for bias to generalize beyond the training data.

We can use the learning curves to diagnose bias or variance problems with our models. For example:

\index{Under-fit}
\index{Over-fit}
\begin{itemize}
	\item \textbf{Model Under-fit}: We may have high-bias in our model if error remains high on both our test and training datasets.
	\item \textbf{Model Over-fit}: We may have high-variance if error on the training dataset is low and the error is high on the test dataset (the example described above).
\end{itemize}

Knowledge of whether your model has a bias or a variance problem can provide some suggestion for improving performance. For example:

\begin{itemize}
	\item \textbf{High-Bias Model}: Add more features to the training set or add additional derived features.
	\item \textbf{High-Variance Model}: Try adding more training examples or train the model on fewer features.
\end{itemize}

I most cases, one can fine-tune the behavior of a model by modifying its parameters which most commonly results in adjusting its resulting bias/variance trade-off. Complex models with lots of parameters commonly have a low bias and perform better with a long training time and/or a lot of training data.
\putbib\end{bibunit}
\newpage\begin{bibunit}% The Clever Algorithms Project: http://www.CleverAlgorithms.com
% (c) Copyright 2011 Jason Brownlee. Some Rights Reserved. 
% This work is licensed under a Creative Commons Attribution-Noncommercial-Share Alike 2.5 Australia License.

% Book Organization
\section{Book Organization} 
\label{intro:organization}
% overview
The remainder of this book is organized into two parts: \emph{Algorithms} that describes a large number of techniques in a complete and a consistent manner presented in a rough algorithm groups, and \emph{Extensions} that reviews more advanced topics suitable for when a number of algorithms have been mastered.

% 
% Algorithms
%
\subsection{Algorithms}
\index{Clever Algorithms!Taxonomy}
% taxonomy
Algorithms are presented in eight groups distilled from the broader fields of study each in their own chapter, as follows: 

\begin{itemize}
	\item \emph{Optimization}: Finding the minimum or maximum of a function, a process that commonly represents the core of a Machine Learning method.
	\item \emph{Regression}: Model relationships between variables using coefficients and a fitting process.
	\item \emph{Discriminant Functions}:	Model relationships between variables to characterize categorical dependent variable.
	\item \emph{Workhorse Methods}: Methods that have become the work horse methods of modern machine learning.
	\item \emph{Dimensionality Reduction}: Methods for projecting high-dimensional data into lower dimensions for clustering, feature selection and visualization.
	\item \emph{Clustering}: Methods for organizing unstructured data into like-groups.
	\item \emph{Model Selection}: Methods search for the best subset of data features and model between the relationships at the same time.
	\item \emph{Ensembles}:	Methods that combine and model the predictions from multiple models.
\end{itemize}

Algorithms were chosen to provide a diverse, interesting, and useful snapshot of the field of statistical Machine  Learning. As such, many algorithms and even classes of algorithms were not covered in this text. The following lists some of these general areas that may be taken as research projects for the keen reader:

\begin{itemize}
	\item \emph{Text Processing}: Methods used for processing documents of text and information retrieval such as Latent Semantic Analysis (LSA) and Latent Dirichlet Allocation (LDA).
	\item \emph{Recommender Systems}: Methods that are popular for use in recommender systems such as Singular Value Decomposition (SVD).
	\item \emph{Graphical Models}: Methods that are common for modeling the relationships between variables such as Hidden Markov Model (HMM), and Bayesian Networks.
	\item \emph{Metaheuristics}: Methods, typically nature-inspired used for optimization and modeling such as ARtificial Neural Networks (ANN) and Genetic Algorithms (GA).
	\item \emph{Reinforcement Learning}: Methods for learning in uncertain environments such as TEmporal Difference Learning (TDL) and Q-learning.
\end{itemize}

% requirements
\index{Clever Algorithms!Template}
A given algorithm is more than just a procedure or code listing, each approach is an island of research. The meta-information that define the context of a technique is just as important to understanding and application as abstract recipes and concrete implementations. A standardized algorithm description is adopted to provide a consistent presentation of algorithms with a mixture of softer narrative descriptions, programmatic descriptions both abstract and concrete, and most importantly useful sources for finding out more information about the technique.

% template
The standardized algorithm description template covers the following subjects:
\begin{itemize}
	\item \emph{Name}: The algorithm name defines the canonical name used to refer to the technique, in addition to common aliases, abbreviations, and acronyms. The name is used as the heading of an algorithm description.
	\item \emph{Taxonomy}: The algorithm taxonomy defines where a technique fits into the field, both the specific sub-fields of Computational Intelligence and Biologically Inspired Computation as well as the broader field of Artificial Intelligence. The taxonomy also provides a context for determining the relationships between algorithms.
	\item \emph{Strategy}: The strategy is an abstract description of the computational model. The strategy describes the information processing actions a technique shall take in order to achieve an objective, providing a logical separation between a computational realization (procedure) and an analogous system (metaphor). A given problem solving strategy may be realized as one of a number of specific algorithms or problem solving systems.
	\item \emph{Overview}: The overview provides a snapshot practical understanding of the algorithm in terms if its \emph{capabilities}, \emph{heuristics}, and \emph{limitations}.
	 \begin{itemize}
			\item \emph{Capabilities}: The capabilities of the method in terms of its suitability toward problem types and practical concerns.
			\item \emph{Heuristics}: The practical usage information for using the technique for problem solving. 
			\item \emph{Limitations}: The constraints and known limitations of a method, typically highlighting the motivations that lead to extensions and development of additional methods.
		\end{itemize}
	\item \emph{Code Listing}: The code listing description provides a minimal but functional version of the technique implemented with a programming language. The code description can be typed into a computer and provide a working execution of the technique. The technique implementation also includes a minimal problem instance to which it is applied, and both the problem and algorithm implementations are complete enough to demonstrate the techniques procedure. The description is presented as a programming source code listing with a terse introductory summary.
	\item \emph{References}: The references section includes a listing of both primary sources of information about the technique as well as useful introductory sources for novices to gain a deeper understanding of the theory and application of the technique. The description consists of hand-selected reference material including books, peer reviewed conference papers, and journal articles.
\end{itemize}


% sample code
\index{R}
\index{R!Versions}
\index{R!Download}
\index{R!CRAN}
Source code examples are included in the algorithm descriptions, and R was selected for use throughout the book. R is an an open source environment for statistical programming and visualization \cite{RDevelopmentCoreTeam2011}. R was selected because it is the tool of choice for statistical and machine learning work in academia and in the industry. A strength of the platform is the large number of third party libraries available that provide plugin-in algorithms and procedures. The Comprehensive R Archive Network (CRAN) provides access to these third party libraries, and one perspective on these libraries is specific tasks or roles, such as machine learning. Finally, R is free to download and use from the Internet.\footnote{R can be downloaded for free from \url{http://www.r-project.org}}

The sample code provides a working version of a given technique for demonstration purposes. Having a tinker with a technique can really bring it to life and provide valuable insight into a method. The sample code is a minimum implementation, providing plenty of opportunity to explore, extend and optimize.
All of the source code for the algorithms presented in this book is available from the companion website, online at \url{http://www.CleverAlgorithms.com}. All algorithm implementations were tested with R version 2.14.0 which should be considered a minimum version to execute the sample scripts. 
For more a high-level introduction into the R environment, see \emph{Appendix A - R: Quick-Start Guide} \ref{ch:appendix1}.

% Advanced Topics
\subsection{Advanced Topics}
%  overview
There are some some advanced topics that cannot be meaningfully considered until one has a firm grasp of a number of algorithms, and these are discussed at the back of the book. 
% examples
The Advanced Topics chapter addresses topics:

\begin{itemize}
	\item \emph{Plotting}: A gentle introduction into plotting in R with examples of common plots for data problems and overview of popular plot types and plot frameworks all with working examples.
	\item \emph{Useful Statistics}: A gentle introduction into applied statistics in R, with a focus on summary statistics and statistical hypothesis tests that may be useful when exploring data sets, all with working examples.
	\item \emph{Model Tuning}: A introduction into methods for tuning a given model to get the most out of it, with usable examples in R.
	\item \emph{Model Verification}: An introduction into model verification such as cross validation and methods to avoid fitting with examples in R.
\end{itemize}

% starting point
Like the background information provided in this chapter, the extensions provide a gentle introduction and starting point into some advanced topics, and references for seeking a deeper understanding.

% How to Read this Book
\subsection{How to Read this Book}
% overview
This book is a reference text that provides a large compendium of algorithm descriptions. 
% how to read it
It is a trusted handbook of practical computational recipes to be consulted when one is confronted with difficult data problem. It is also an encompassing guidebook of modern statistical machine learning methods that may be browsed for inspiration, exploration, and general interest.

% audience
The audience for this work may be interested in the fields of Statistics and Machine Learning and may count themselves as belonging to one of the following broader groups:

\begin{itemize}
	\item \emph{Scientists}: Research scientists concerned with theoretically or empirically investigating algorithms, addressing questions such as: \emph{What is the motivating information processing strategy for a given technique? What are some algorithms that may be used in a comparison within a given subfield or across subfields?}
	\item \emph{Engineers}: Programmers and developers concerned with implementing, applying, or maintaining algorithms, addressing questions such as: \emph{What is the application procedure for a given technique? What are the best practice heuristics for employing a given technique?}
	\item \emph{Students}: Undergraduate and graduate students interested in learning about techniques, addressing questions such as: \emph{What are some interesting algorithms to study? How to play with a given approach?}
	\item \emph{Amateurs}: Practitioners interested in knowing more about algorithms, addressing questions such as: \emph{What classes of techniques exist and what algorithms do they provide? What is the intuition of a given technique?}
\end{itemize}
\putbib\end{bibunit}
\newpage\begin{bibunit}% The Clever Algorithms Project: http://www.CleverAlgorithms.com
% (c) Copyright 2011 Jason Brownlee. Some Rights Reserved. 
% This work is licensed under a Creative Commons Attribution-Noncommercial-Share Alike 2.5 Australia License.

% Further Reading
\section{Further Reading} 
\label{intro:furtherreading}
\index{Further Reading}
% overview
This book is not an introduction to Machine Learning or related sub-fields, nor is it a field guide for a specific class of algorithms. This section provides some pointers to selected books and articles for those readers seeking a deeper understanding of the fields of study to which the Clever Algorithms described in this book belong.

% Artificial Intelligence
\subsection{Artificial Intelligence}
\index{Artificial Intelligence!References}
Artificial Intelligence is large field of study and many excellent texts have been written to introduce the subject. Russell and Novig's ``\emph{Artificial Intelligence: A Modern Approach}'' is an excellent introductory text providing a broad and deep review of what the field has to offer and is useful for students and practitioners alike \cite{Russell2009}. Luger and Stubblefield's ``\emph{Artificial Intelligence: Structures and Strategies for Complex Problem Solving}'' is also an excellent reference text, providing a more empirical approach to the field \cite{Luger1993}.

% Machine Learning
\subsection{Machine Learning}
\index{Machine Learning!References}
Machine Learning has changed a lot over the last 20 years. 
Tom Mitchell's ``\emph{Machine Learning}'' is classic text providing an early shaping of the field \cite{Mitchell1997}.
Duda et~al.\ also provide a classic reference text in ``\emph{Pattern Classification}'' that provides a broader perspective of machine learning \cite{Duda2001}.

Bishop's ``\emph{Pattern Recognition and Machine Learning}'' provides an excellent theoretical study of modern machine learning methods \cite{Bishop2007}.
Hastie et~al.\ text ``\emph{The Elements of Statistical Learning: Data Mining, Inference, and Prediction}'' provides a modern perspective on the field, aggressively seeking sound statistically theoretical understanding of many modern methods \cite{Hastie2009}.

% Data Mining
\subsection{Data Mining}
\index{Data Mining!References}
Data Mining may be taken as a filed concerned with applied Machine Learning. 
Witten and Frank provide an excellent introduction into the field of Data Mining with some practical examples in their renowned open source Data Mining WEKA software \cite{Witten2011}.
% TODO another ref!

% R
\subsection{R}
\index{R!References}
There are many excellent books on R and almost all of them are free and can be found on the R website at \url{http://www.r-project.org}.

Two non-free excellent introductions to R that should be considered by the serious reader include Crawley's ``\emph{The R Book}'' which provides an extensive reference \cite{Crawley2007} and Matloff's ``\emph{The Art of R Programming: A Tour of Statistical Software Design}'' which provides a focus on programing in R \cite{Matloff2011}.

Finally, the true power in R is in the extension packages on CRAN. The best way to get started with CRAN is to look at some `views' that group like-packages under a common theme. Take a look at the \emph{Machine Learning and Statistical Learning}, \emph{Optimization and Mathematical Programming}, and \emph{Cluster Analysis and Finite Mixture Models} views at the CRAN views webpage \url{http://cran.r-project.org/web/views}.
\putbib\end{bibunit}

