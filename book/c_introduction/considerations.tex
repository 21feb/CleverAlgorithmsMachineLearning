% The Clever Algorithms Project: http://www.CleverAlgorithms.com
% (c) Copyright 2011 Jason Brownlee. Some Rights Reserved. 
% This work is licensed under a Creative Commons Attribution-Noncommercial-Share Alike 2.5 Australia License.

% Considerations
\section{Considerations} 
\label{intro:considerations}

overview of machine learning considerations



% Bias-Variance
\subsection{Bias-Variance Trade-off}

bias - approximation is too crude
variance - overfitting training data

typically when the bias is low, variance is high, and vice-versa

each problem has a sweet-spot in this trade-off
i.e. prediction error on the training set goes down, but prediction error on the test set goes down then back up, you want the bottom of that inflection point.

this has to do with model selection - which model to use

cross validation addresses this problem - average fit over each fold gives a better idea of performance.

voting an ensemble methods can reduce variance

discussion of bias and variance in the context of ensembles \cite{Dietterich1995}
- bias as the assumptions made by a method
- absolute bias - (what area in the space of possible models) assumption that target function comes from a class
- relative bias - (what preferences in the area) target function is more likely to come from one set of functions than another
must adopt a bias to generalize beyond the training data

statistical meaning of bias is the error is expected to make in a dataset. - systematic error

variance is squared difference between a model and the averaged model (variance in the model between source datasets?)

minimize statistical bias and variance = reduce error


kind of related, but also about bagging \cite{Tibshirani1996a}

model status
- underfit - bias - too general
- fit
- overfit - high-bias - too specific

Are you underfit or overfit? 
- bias: train and validation set have high error
- variance - train is low error, validation is high

diagnosis (learning curves)
- large variance - get more data
- large gap between training error and validation error - get more data
- high error and small gap between train and validation error - high bias


Improving your model
Andrew Ng from Stanford
- getting more training examples (fix high variance)
- feature selection - train on less features (fix high variance, high bias - fewer features not helpful)
- collect additional features (fix high bias)
- adding derived features (polynomial features) (fix high bias)
- modifying parameters of algorithm (model or regularized model, say lambda) (fix high bias, fix high variance)


% Methodology
\subsection{Methodology}

Methodology: (Andrew Ng)
prepare model on test set
select parameters on the validation set
test on the test set

use learning curves to diagnose bias/variance problems


Approach (andrew ng)
- Brainstorm and generate lots of approaches to the problem - sample each to get an idea of where to spend time...
- try simple algorithms first - something quick and dirty
- plot learning curves
	- avoid premature optimization (use evidence to guide project)
- perform error analysis
	- manually examine errors
	- look for systematic patterns in mis-classified instances

use precision and recall
- tune threshold to trade-off precision and recall
	- higher threshold = higher precision, lower recall
	- lower threshold = lower precision, higher recall
- use f score

some algorithm work better with more data
- complex model - train error will be small (low bias) and lots of data and unlikely to over fit - train and test error will be similar (low variance)
- can human experts do it - we should have enough to make predictions

