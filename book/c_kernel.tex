% The Clever Algorithms Project: http://www.CleverAlgorithms.com
% (c) Copyright 2010 Jason Brownlee. Some Rights Reserved. 
% This work is licensed under a Creative Commons Attribution-Noncommercial-Share Alike 2.5 Australia License.

% This is a chapter

\renewcommand{\bibsection}{\subsection{\bibname}}
\begin{bibunit}

\chapter{Kernel Machines}
\label{ch:kernels}
\index{Kernel Machines}

\section{Overview}
This chapter describes Kernel Machines.


% Strategy: Problem solving plan
% The strategy is an abstract description of the computational model. The strategy describes the information processing actions a technique shall take in order to achieve an objective. The strategy provides a logical separation between a computational realization (procedure) and a analogous system (metaphor). A given problem solving strategy may be realized as one of a number specific algorithms or problem solving systems. The strategy description is textual using information processing and algorithmic terminology.
\subsection{Strategy}
% What is the information processing objective of a technique?
% What is a techniques plan of action?

don't have to use polyominals to enlarge the feature space
some great finding about relating the feature spaces...
reproducing Hilbert spaces...?
start with kernel - generate spaces
Radial basis functions - popular one is a bump - essentially infinite basis functions

methods:
kernel density
kernel smoothing

Kernel Machines... generate function spaces
Any model that used to be a linear model - can be expressed as a kernel

use special kernels for domain - closeness - smoothness - invariance in the domain
	- like string kernels - distance between genome sequences


computational costs
no automatic way of doing feature selection - keeps all the features you provide
	- if you're sparse you're going to have problems



% Heuristics: Usage guidelines
% The heuristics element describe the commonsense, best practice, and demonstrated rules for applying and configuring a parameterized algorithm. The heuristics relate to the technical details of the techniques procedure and data structures for general classes of application (neither specific implementations not specific problem instances). The heuristics are described textually, such as a series of guidelines in a bullet-point structure.
\subsection{Heuristics}
% What are the suggested configurations for a technique?
% What are the guidelines for the application of a technique to a problem instance?

\begin{itemize}
	\item 
\end{itemize}



% References: Deeper understanding
% The references element description includes a listing of both primary sources of information about the technique as well as useful introductory sources for novices to gain a deeper understanding of the theory and application of the technique. The description consists of hand-selected reference material including books, peer reviewed conference papers, journal articles, and potentially websites. A bullet-pointed structure is suggested.
\subsection{References}
% What are the primary sources for a technique?
% What are the suggested reference sources for learning more about a technique?

% primary sources
\subsubsection{Primary Sources}


% more info
\subsubsection{More Information}



\putbib
\end{bibunit}


\newpage\begin{bibunit}% The Clever Algorithms Project: http://www.CleverAlgorithms.com
% (c) Copyright 2011 Jason Brownlee. Some Rights Reserved. 
% This work is licensed under a Creative Commons Attribution-Noncommercial-Share Alike 2.5 Australia License.

% Name
% The algorithm name defines the canonical name used to refer to the technique, in addition to common aliases, abbreviations, and acronyms. The name is used in terms of the heading and sub-headings of an algorithm description.
\section{Support Vector Machines} 
\label{sec:svm}
\index{SVM}
\index{Support Vector Machines}

% other names
% What is the canonical name and common aliases for a technique?
% What are the common abbreviations and acronyms for a technique?
\emph{Support Vector Machines, SVM}

% Taxonomy: Lineage and locality
% The algorithm taxonomy defines where a techniques fits into the field, both the specific subfields of Computational Intelligence and Biologically Inspired Computation as well as the broader field of Artificial Intelligence. The taxonomy also provides a context for determining the relation- ships between algorithms. The taxonomy may be described in terms of a series of relationship statements or pictorially as a venn diagram or a graph with hierarchical structure.
\subsection{Taxonomy}
% To what fields of study does a technique belong?
% What are the closely related approaches to a technique?
Support Vector Machines are a kernel method.

% Strategy: Problem solving plan
% The strategy is an abstract description of the computational model. The strategy describes the information processing actions a technique shall take in order to achieve an objective. The strategy provides a logical separation between a computational realization (procedure) and a analogous system (metaphor). A given problem solving strategy may be realized as one of a number specific algorithms or problem solving systems. The strategy description is textual using information processing and algorithmic terminology.
\subsection{Strategy}
% What is the information processing objective of a technique?
% What is a techniques plan of action?

try to create a classifier directly, not a probability of a class like logit
find hyper plane that maximizes the margin
only works if the training data is separable
it is a linear classifier
support points... boundary of the margin

use soft margin if the data is not linearly separable - budget of max overlap allowed
budget - a regularization parameter

basis transformations - increase dimensions using polynominals
- use both soft margin and more dimensions

radial basis svm...
gamma is like the steepness of the kernel

use crossvalidation to refine the parameter - b
algorithm for finding best for all values of b
regularization



% help me use this technique
\subsection{Overview}

% what it is good at
\subsubsection{Features}

\begin{itemize}
	\item 
\end{itemize}

% what it is not good at
\subsubsection{Limitations}

\begin{itemize}
	\item 
\end{itemize}


% sample script in R
\subsection{Code Listing}
% listing
Listing~\ref{e1071_support_vector_machines} provides a code listing of the Support Vector Machines algorithm in R to classify examples from a three-dimensional dataset.
% algorithm and package
The example uses the \texttt{svm()} function in the \texttt{e1071} package \cite{Meyer2011}.
% TODO more about the package and libsvm

% problem
The test problem is a three-dimensional classification problem, where the $x$ and $y$ attributes are numerical and drawn from normal distributions around 0 and 4. A class value of 0 or 1 is assigned to each coordinate such that the two classes can be separated by a straight line. The dataset is split into a training set to make the model comprised of 67\% of the samples, and a test set for assessing the model comprised of 33\% of the samples.

% code listing
\lstinputlisting[firstline=7,language=r,caption={Example of Support Vector Machines in R using the \texttt{svm()} function of the \texttt{e1071} package.}, label=e1071_support_vector_machines]{../src/algorithms/kernel/e1071_support_vector_machines.R}

% other packages
Karatzoglou, Meyer, and Hornik provide a comprehensive overview of the R packages that provide SVM implementations, providing a summary of the capabilities and script examplse of each \cite{Karatzoglou2006}.


% References: Deeper understanding
% The references element description includes a listing of both primary sources of information about the technique as well as useful introductory sources for novices to gain a deeper understanding of the theory and application of the technique. The description consists of hand-selected reference material including books, peer reviewed conference papers, journal articles, and potentially websites. A bullet-pointed structure is suggested.
\subsection{References}
% What are the primary sources for a technique?
% What are the suggested reference sources for learning more about a technique?

% primary sources
\subsubsection{Primary Sources}



% more info
\subsubsection{More Information}



% END
\putbib\end{bibunit}
