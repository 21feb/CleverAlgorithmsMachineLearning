% The Clever Algorithms Project: http://www.CleverAlgorithms.com
% (c) Copyright 2013 Jason Brownlee. Some Rights Reserved. 
% This work is licensed under a Creative Commons Attribution-Noncommercial-Share Alike 2.5 Australia License.

% This is a chapter

\renewcommand{\bibsection}{\subsection{\bibname}}
\begin{bibunit}

\chapter{Decision Trees}
\label{ch:trees}
\index{Decision Trees}
\index{Trees}

\section{Overview}
This chapter describes Decision Trees.


% Strategy: Problem solving plan
% The strategy is an abstract description of the computational model. The strategy describes the information processing actions a technique shall take in order to achieve an objective. The strategy provides a logical separation between a computational realization (procedure) and a analogous system (metaphor). A given problem solving strategy may be realized as one of a number specific algorithms or problem solving systems. The strategy description is textual using information processing and algorithmic terminology.
\subsection{Strategy}
% What is the information processing objective of a technique?
% What is a techniques plan of action?

grow a tree from data

prune a tree using cost-complexity graph, (alpha is a meta parameter that controls the degree of stabilization.)

handle huge data sets - divide conquer 
handle mixed features
ignore redundant variables
small are easy to under stand
large are hard to understand
prediction performance is very poor

use cost complexity pruning


trees are made by recursive partitioning 


% Heuristics: Usage guidelines
% The heuristics element describe the commonsense, best practice, and demonstrated rules for applying and configuring a parameterized algorithm. The heuristics relate to the technical details of the techniques procedure and data structures for general classes of application (neither specific implementations not specific problem instances). The heuristics are described textually, such as a series of guidelines in a bullet-point structure.
\subsection{Heuristics}
% What are the suggested configurations for a technique?
% What are the guidelines for the application of a technique to a problem instance?

\begin{itemize}
	\item Trees are fast to train (simple growing algorithm).
	\item Trees can handle missing values (treat missing as a value, surrogate splits)
	\item Few tunable parameters
	\item Trees can explain why they make their decisions, the tree model can be understood by subject matter experts.
	\item Trees cannot addres curved decision surfaces at all or very well.
	\item Trees generally have hard or blocky decision surfaces.
	\item Trees generally perform poorly across a range of problems.


	% limitations
	\item Need a lot of data to have a lot of splits
	\item Can run out of data after only a few splits in high-dimensions
	\item Each split reduces the data available for subsequent splits (data fragmentation)
	\item Performs poorly if the target function has many dependent variables 
	\item high variance - small changes in training data result in very different trees
	\item errors / decisions made early propagate through the rest of the tree
	\item pruning is important

\end{itemize}



% References: Deeper understanding
% The references element description includes a listing of both primary sources of information about the technique as well as useful introductory sources for novices to gain a deeper understanding of the theory and application of the technique. The description consists of hand-selected reference material including books, peer reviewed conference papers, journal articles, and potentially websites. A bullet-pointed structure is suggested.
\subsection{References}
% What are the primary sources for a technique?
% What are the suggested reference sources for learning more about a technique?

% primary sources
\subsubsection{Primary Sources}


% more info
\subsubsection{More Information}



\putbib
\end{bibunit}

\newpage\begin{bibunit}% The Clever Algorithms Project: http://www.CleverAlgorithms.com
% (c) Copyright 2013 Jason Brownlee. Some Rights Reserved. 
% This work is licensed under a Creative Commons Attribution-Noncommercial-Share Alike 2.5 Australia License.


% Name
% The algorithm name defines the canonical name used to refer to the technique, in addition to common aliases, abbreviations, and acronyms. The name is used in terms of the heading and sub-headings of an algorithm description.
\section{Classification And Regression Tree}
\label{sec:cart}
\index{CART}
\index{Classification And Regression Tree}

% other names
% What is the canonical name and common aliases for a technique?
% What are the common abbreviations and acronyms for a technique?
\emph{Classification And Regression Tree, CART.}

% Taxonomy: Lineage and locality
% The algorithm taxonomy defines where a techniques fits into the field, both the specific subfields of Computational Intelligence and Biologically Inspired Computation as well as the broader field of Artificial Intelligence. The taxonomy also provides a context for determining the relation- ships between algorithms. The taxonomy may be described in terms of a series of relationship statements or pictorially as a venn diagram or a graph with hierarchical structure.
\subsection{Taxonomy}
% To what fields of study does a technique belong?
% What are the closely related approaches to a technique?
The Classification And Regression Tree or CART is a Decision Tree algorithm. CART is a registered trademark.

% Strategy: Problem solving plan
% The strategy is an abstract description of the computational model. The strategy describes the information processing actions a technique shall take in order to achieve an objective. The strategy provides a logical separation between a computational realization (procedure) and a analogous system (metaphor). A given problem solving strategy may be realized as one of a number specific algorithms or problem solving systems. The strategy description is textual using information processing and algorithmic terminology.
\subsection{Strategy}
% What is the information processing objective of a technique?
% What is a techniques plan of action?
A textual description of the information processing strategy.



% help me use this technique
\subsection{Overview}

% what it is good at
\subsubsection{Features}

\begin{itemize}
	\item The algorithm is non-parametric.
	\item The algorithms can be used to create classification and regression trees.
\end{itemize}

% what it is not good at
\subsubsection{Limitations}

\begin{itemize}
	\item 
\end{itemize}




% sample script in R
\subsection{Code Listing}
% listing
Listing~\ref{rpart_cart} provides a code listing of the CART method in R to classify examples from a three-dimensional dataset.
% algorithm and package
The example uses the \texttt{rpart()} function (recursive partitioning) in the \texttt{rpart} package \cite{Meyer2011}.
% TODO more about the package and the functions capabilities

% problem
The test problem is a three-dimensional classification problem, where the $x$ and $y$ attributes are numerical and drawn from normal distributions around 0 and 4. A class value of 0 or 1 is assigned to each coordinate such that the two classes can be separated by a straight line. The dataset is split into a training set to make the model comprised of 67\% of the samples, and a test set for assessing the model comprised of 33\% of the samples.

% TODO provide prediction example

% code listing
\lstinputlisting[firstline=7,language=r,caption={Example of CART in R using the \texttt{rpart()} function of the \texttt{rpart} package.}, label=rpart_cart]{../src/algorithms/trees/rpart_cart.R}

% other packages
Other packages provide implementations of the CART method, such as the \texttt{tree} package.


% References: Deeper understanding
% The references element description includes a listing of both primary sources of information about the technique as well as useful introductory sources for novices to gain a deeper understanding of the theory and application of the technique. The description consists of hand-selected reference material including books, peer reviewed conference papers, journal articles, and potentially websites. A bullet-pointed structure is suggested.
\subsection{References}
% What are the primary sources for a technique?
% What are the suggested reference sources for learning more about a technique?

% primary sources
\subsubsection{Primary Sources}

Seminal reference is the book \cite{Breiman1984}.

% more info
\subsubsection{More Information}



% END
\putbib\end{bibunit}
\newpage\begin{bibunit}% The Clever Algorithms Project: http://www.CleverAlgorithms.com
% (c) Copyright 2013 Jason Brownlee. Some Rights Reserved. 
% This work is licensed under a Creative Commons Attribution-Noncommercial-Share Alike 2.5 Australia License.


% Name
% The algorithm name defines the canonical name used to refer to the technique, in addition to common aliases, abbreviations, and acronyms. The name is used in terms of the heading and sub-headings of an algorithm description.
\section{C4.5} 
\label{sec:c4.5}
\index{C4.5}

% other names
% What is the canonical name and common aliases for a technique?
% What are the common abbreviations and acronyms for a technique?
\emph{C4.5.}

% Taxonomy: Lineage and locality
% The algorithm taxonomy defines where a techniques fits into the field, both the specific subfields of Computational Intelligence and Biologically Inspired Computation as well as the broader field of Artificial Intelligence. The taxonomy also provides a context for determining the relation- ships between algorithms. The taxonomy may be described in terms of a series of relationship statements or pictorially as a venn diagram or a graph with hierarchical structure.
\subsection{Taxonomy}
% To what fields of study does a technique belong?
% What are the closely related approaches to a technique?
C4.5 is a Decision Tree algorithm.

% Strategy: Problem solving plan
% The strategy is an abstract description of the computational model. The strategy describes the information processing actions a technique shall take in order to achieve an objective. The strategy provides a logical separation between a computational realization (procedure) and a analogous system (metaphor). A given problem solving strategy may be realized as one of a number specific algorithms or problem solving systems. The strategy description is textual using information processing and algorithmic terminology.
\subsection{Strategy}
% What is the information processing objective of a technique?
% What is a techniques plan of action?
A textual description of the information processing strategy.

% Heuristics: Usage guidelines
% The heuristics element describe the commonsense, best practice, and demonstrated rules for applying and configuring a parameterized algorithm. The heuristics relate to the technical details of the techniques procedure and data structures for general classes of application (neither specific implementations not specific problem instances). The heuristics are described textually, such as a series of guidelines in a bullet-point structure.
\subsection{Heuristics}
% What are the suggested configurations for a technique?
% What are the guidelines for the application of a technique to a problem instance?

\begin{itemize}
	\item todo
\end{itemize}


J4.8 from WEKA provides a version of C4.5


% References: Deeper understanding
% The references element description includes a listing of both primary sources of information about the technique as well as useful introductory sources for novices to gain a deeper understanding of the theory and application of the technique. The description consists of hand-selected reference material including books, peer reviewed conference papers, journal articles, and potentially websites. A bullet-pointed structure is suggested.
\subsection{References}
% What are the primary sources for a technique?
% What are the suggested reference sources for learning more about a technique?

% primary sources
\subsubsection{Primary Sources}

Book \cite{Quinlan1993}.

% more info
\subsubsection{More Information}



% END
\putbib\end{bibunit}
